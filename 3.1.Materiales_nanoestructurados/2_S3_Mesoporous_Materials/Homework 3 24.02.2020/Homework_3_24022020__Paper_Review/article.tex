\documentclass[11pt]{article}
\makeatletter\if@twocolumn\PassOptionsToPackage{switch}{lineno}\else\fi\makeatother

      \makeatletter
\usepackage{wrapfig}
\newcounter{aubio}

\long\def\bioItem{%
\@ifnextchar[{\@bioItem}{\@@bioItem}}

\long\def\@bioItem[#1]#2#3{
 \stepcounter{aubio}
 \expandafter\gdef\csname authorImage\theaubio\endcsname{#1}
 \expandafter\gdef\csname authorName\theaubio\endcsname{#2}
 \expandafter\gdef\csname authorDetails\theaubio\endcsname{#3}
}

\long\def\@@bioItem#1#2{
 \stepcounter{aubio}
 \expandafter\gdef\csname authorName\theaubio\endcsname{#1}
 \expandafter\gdef\csname authorDetails\theaubio\endcsname{#2}
}

\newcommand{\checkheight}[1]{%
  \par \penalty-100\begingroup%
  \setbox8=\hbox{#1}%
  \setlength{\dimen@}{\ht8}%
  \dimen@ii\pagegoal \advance\dimen@ii-\pagetotal
  \ifdim \dimen@>\dimen@ii
    \break
  \fi\endgroup}

\def\printBio{%
  \@tempcnta=0
   \loop
     \advance \@tempcnta by 1
     \def\aubioCnt{\the\@tempcnta}
     \setlength{\intextsep}{0pt}%
     \setlength{\columnsep}{10pt}%
     \newbox\boxa%
     \setbox\boxa\vbox{\csname authorDetails\aubioCnt\endcsname}
     \expandafter\ifx\csname authorImage\aubioCnt\endcsname\relax%
      \else%
       \checkheight{\includegraphics[height=1.25in,width=1in,keepaspectratio]{\csname authorImage\aubioCnt\endcsname}}
        \begin{wrapfigure}{l}{25mm}
         \includegraphics[height=1.25in,width=1in,keepaspectratio]{\csname authorImage\aubioCnt\endcsname}%height=145pt
        \end{wrapfigure}\par
      \fi
     {\parindent0pt\textbf{\csname authorName\aubioCnt\endcsname}\csname authorDetails\aubioCnt\endcsname \par\bigskip%
     \expandafter\ifx\csname authorImage\aubioCnt\endcsname\relax\else%
      \ifdim\the\ht\boxa < 90pt\vskip\dimexpr(90pt -\the\ht\boxa-1pc)\fi%
     \fi}%for adding additional vskip for avoiding image overlap.
      \ifnum\@tempcnta < \theaubio
   \repeat
   }

\makeatother

      



\usepackage{amsfonts,amssymb,amsbsy,latexsym,amsmath,tabulary,graphicx,times,xcolor}
\usepackage[utf8x]{inputenc}
\usepackage{fancyhdr}
\def\NormalBaseline{\def\baselinestretch{1.1}}
\makeatletter
\def\hlinewd#1{%
  \noalign{\ifnum0=`}\fi\hrule \@height #1%
  \futurelet\reserved@a\@xhline}
\def\tbltoprule{\hlinewd{.8pt}}%\\[-10pt]}
\def\tblbottomrule{\hlinewd{.8pt}}
\def\tblmidrule{\hline\noalign{\vspace*{2pt}}}

\def\@shorttitle{\@empty}
\def\shorttitle#1{\gdef\@shorttitle{#1}}

\fancypagestyle{custom}{
\fancyhf{}
\fancyhead[C]{\@shorttitle}
\fancyhead[R]{\thepage}
\fancyfoot[C]{}
\renewcommand\headrulewidth{0.4pt}
\renewcommand\footrulewidth{0pt}
}
\fancypagestyle{plain}{
\fancyhf{}
\renewcommand\headrulewidth{0.4pt}
}


\makeatother

\usepackage{times}

\usepackage[a4paper,margin=2.5cm,headsep=.7cm,headheight=18pt,top=3cm,footnotesep=1.5\baselineskip]{geometry}
\usepackage{caption}
\captionsetup[figure]{labelfont=bf,labelsep=newline,justification=centerlast,labelfont={small,sc,bf},font=small,aboveskip=.3\baselineskip}

\captionsetup[table]{labelfont=bf,labelsep=newline,justification=centerlast,labelfont={small,sc,bf},font=small,aboveskip=.3\baselineskip}
\linespread{1.5}

\setcounter{totalnumber}{4}
\def\topfraction{0.9}
\def\bottomfraction{0.4}
\def\floatpagefraction{0.8}
\def\textfraction{0.1}
\widowpenalty 10000
\clubpenalty 10000
\makeatletter
\setlength\intextsep   {1.5\baselineskip \@plus 2\p@ \@minus 2\p@}
\makeatother

  
%%%%%%%%%%%%%%%%%%%%%%%%%%%%%%%%%%%%%%%%%%%%%%%%%%%%%%%%%%%%%%%%%%%%%%%%%%
% Following additional macros are required to function some 
% functions which are not available in the class used.
%%%%%%%%%%%%%%%%%%%%%%%%%%%%%%%%%%%%%%%%%%%%%%%%%%%%%%%%%%%%%%%%%%%%%%%%%%
\usepackage{url,multirow,morefloats,floatflt,cancel,tfrupee}
\makeatletter


\AtBeginDocument{\@ifpackageloaded{textcomp}{}{\usepackage{textcomp}}}
\makeatother
\usepackage{colortbl}
\usepackage{xcolor}
\usepackage{pifont}
\usepackage[nointegrals]{wasysym}
\urlstyle{rm}
\makeatletter

%%%For Table column width calculation.
\def\mcWidth#1{\csname TY@F#1\endcsname+\tabcolsep}

%%Hacking center and right align for table
\def\cAlignHack{\rightskip\@flushglue\leftskip\@flushglue\parindent\z@\parfillskip\z@skip}
\def\rAlignHack{\rightskip\z@skip\leftskip\@flushglue \parindent\z@\parfillskip\z@skip}

%Etal definition in references
\@ifundefined{etal}{\def\etal{\textit{et~al}}}{}


%\if@twocolumn\usepackage{dblfloatfix}\fi
\usepackage{ifxetex}
\ifxetex\else\if@twocolumn\@ifpackageloaded{stfloats}{}{\usepackage{dblfloatfix}}\fi\fi

\AtBeginDocument{
\expandafter\ifx\csname eqalign\endcsname\relax
\def\eqalign#1{\null\vcenter{\def\\{\cr}\openup\jot\m@th
  \ialign{\strut$\displaystyle{##}$\hfil&$\displaystyle{{}##}$\hfil
      \crcr#1\crcr}}\,}
\fi
}

%For fixing hardfail when unicode letters appear inside table with endfloat
\AtBeginDocument{%
  \@ifpackageloaded{endfloat}%
   {\renewcommand\efloat@iwrite[1]{\immediate\expandafter\protected@write\csname efloat@post#1\endcsname{}}}{\newif\ifefloat@tables}%
}%

\def\BreakURLText#1{\@tfor\brk@tempa:=#1\do{\brk@tempa\hskip0pt}}
\let\lt=<
\let\gt=>
\def\processVert{\ifmmode|\else\textbar\fi}
\let\processvert\processVert

\@ifundefined{subparagraph}{
\def\subparagraph{\@startsection{paragraph}{5}{2\parindent}{0ex plus 0.1ex minus 0.1ex}%
{0ex}{\normalfont\small\itshape}}%
}{}

% These are now gobbled, so won't appear in the PDF.
\newcommand\role[1]{\unskip}
\newcommand\aucollab[1]{\unskip}
  
\@ifundefined{tsGraphicsScaleX}{\gdef\tsGraphicsScaleX{1}}{}
\@ifundefined{tsGraphicsScaleY}{\gdef\tsGraphicsScaleY{.9}}{}
% To automatically resize figures to fit inside the text area
\def\checkGraphicsWidth{\ifdim\Gin@nat@width>\linewidth
	\tsGraphicsScaleX\linewidth\else\Gin@nat@width\fi}

\def\checkGraphicsHeight{\ifdim\Gin@nat@height>.9\textheight
	\tsGraphicsScaleY\textheight\else\Gin@nat@height\fi}

\def\fixFloatSize#1{}%\@ifundefined{processdelayedfloats}{\setbox0=\hbox{\includegraphics{#1}}\ifnum\wd0<\columnwidth\relax\renewenvironment{figure*}{\begin{figure}}{\end{figure}}\fi}{}}
\let\ts@includegraphics\includegraphics

\def\inlinegraphic[#1]#2{{\edef\@tempa{#1}\edef\baseline@shift{\ifx\@tempa\@empty0\else#1\fi}\edef\tempZ{\the\numexpr(\numexpr(\baseline@shift*\f@size/100))}\protect\raisebox{\tempZ pt}{\ts@includegraphics{#2}}}}

%\renewcommand{\includegraphics}[1]{\ts@includegraphics[width=\checkGraphicsWidth]{#1}}
\AtBeginDocument{\def\includegraphics{\@ifnextchar[{\ts@includegraphics}{\ts@includegraphics[width=\checkGraphicsWidth,height=\checkGraphicsHeight,keepaspectratio]}}}

\DeclareMathAlphabet{\mathpzc}{OT1}{pzc}{m}{it}

\def\URL#1#2{\@ifundefined{href}{#2}{\href{#1}{#2}}}

%%For url break
\def\UrlOrds{\do\*\do\-\do\~\do\'\do\"\do\-}%
\g@addto@macro{\UrlBreaks}{\UrlOrds}



\edef\fntEncoding{\f@encoding}
\def\EUoneEnc{EU1}
\makeatother
\def\floatpagefraction{0.8} 
\def\dblfloatpagefraction{0.8}
\def\style#1#2{#2}
\def\xxxguillemotleft{\fontencoding{T1}\selectfont\guillemotleft}
\def\xxxguillemotright{\fontencoding{T1}\selectfont\guillemotright}

\newif\ifmultipleabstract\multipleabstractfalse%
\newenvironment{typesetAbstractGroup}{}{}%

%%%%%%%%%%%%%%%%%%%%%%%%%%%%%%%%%%%%%%%%%%%%%%%%%%%%%%%%%%%%%%%%%%%%%%%%%%

\usepackage{natbib}




\usepackage{titlesec}
\usepackage[T1]{fontenc}
\setcounter{secnumdepth}{5}
 
\titleformat{\section}[hang]{\NormalBaseline\filright\large\bfseries}
{\large\thesection}
{10pt}
{}
[]
\titleformat{\subsection}[hang]{\NormalBaseline\filright\bfseries}
{\thesubsection}
{10pt}
{}
[]
\titleformat{\subsubsection}[hang]{\NormalBaseline\filright\bfseries\itshape}
{\upshape\thesubsubsection}
{10pt}
{}
[]
\titleformat{\paragraph}[runin]{\NormalBaseline\filright\bfseries}
{\theparagraph}
{10pt}
{}
[]
\titleformat{\subparagraph}[runin]{\NormalBaseline\filright\bfseries\itshape}
{\thesubparagraph}
{10pt}
{}
[]

\titlespacing{\section}{0pt}{1.5\baselineskip}{.2\baselineskip}  
\titlespacing{\subsection}{0pt}{1.5\baselineskip}{.2\baselineskip}  
\titlespacing{\subsubsection}{0pt}{1.5\baselineskip}{.2\baselineskip}  
\titlespacing{\paragraph}{0pt}{.5\baselineskip}{10pt}  
\titlespacing{\subparagraph}{0pt}{.5\baselineskip}{10pt}  
  

  




%%%%%%%%%%%%%%%%%%%%%%%%%%%%%%%%%%%%%%%%%%
% Feature enabled:
%secnum: unnumbered
%linespacing: 1.25
%%%%%%%%%%%%%%%%%%%%%%%%%%%%%%%%%%%%%%%%%%

\setcounter{secnumdepth}{0}

\linespread{1.25}

\begin{document}



\renewcommand*\rmdefault{bch}\normalfont\upshape

\shorttitle{Surface Tension \& Surface Energy}

\date{}  

  
\title{\NormalBaseline\raggedright\bfseries Homework \#3 24.02.2020 {\textemdash} Paper Review}
  \let\origthanks\thanks
\renewcommand\thanks[1]{\begingroup\let\rlap\relax\origthanks{#1}\endgroup}
\author{\hskip2pc\parbox{.95\textwidth}{\bfseries\large Antonio Osamu Katagiri Tanaka\textsuperscript{1}\thanks{E-mail: A01212611@itesm.mx}
      \\[3pt] 
    % Address
    \normalfont\itshape\NormalBaseline \textsuperscript{\upshape 1} 
    ITESM\unskip, \normalfont\itshape\NormalBaseline Av. Eugenio Garza Sada 2501 Sur\unskip, N.L.\unskip, Monterrey\unskip, Mexico}}
    
    
\maketitle 
\pagestyle{custom}

    
\section{Surface Tension, Surface Energy, and Chemical Potential Due to Their Difference \unskip~\protect\cite{691550:16479843}}
In the nanoscale, atoms and molecules have a significant impact on a system's properties as they constitute most of the system surfaces. Surface energy is hence a required to be studied to understand and control nanostructured materials. Moreover, surface geometry has an effect on the chemical potential of the system, as listed below\unskip~\cite{691550:16484266}.

Nanosystems have very large surface extensions compared to bulk systems. "Based on geometrical considerations, it is easy to show that once the total amount of material is fixed, the total surface area of an ensemble of objects in inversely proportional to the linear size of the object, and so will be the contribution of the energy of the surface to the total energy system.\unskip~\cite{691550:16484266}" As the amount of atoms that are exposed at the surface increases, typically nanosystems undergo a number of re-arrangements in order to reduce the energy of the surface, for instance: a) Surface relaxation {\textemdash} as atoms are bounded to find a different equilibrium position. b) Reconstruction {\textemdash} where new bonds form at the surface and the geometry changes. c) Aggregating {\textemdash} as structures agglomerate to one another, reducing the surface area. d) Ostwald ripening {\textemdash} as larger particles grow at the expenses of smaller particles\unskip~\cite{691550:16484266}.

The addition of even one atom of molecule to the material, does change the surface area. The energy change due to the addition of a single atom to the particle, must be equal to the energy change associated to the change in surface area. One of the most important effects of the coupling between size and chemical potential is the dramatic drop of the melting temperature with size\unskip~\cite{691550:16484266}.

As stated by Lughi et al., surface tension and surface energy plays a significant role in nanostructured materials, which can be amended by time or kinetic processes\unskip~\cite{691550:16484266}. In the past, Shuttleworth \unskip~\cite{691550:16484445} described the relationship between surface tension and surface energy. However, Shuttleworth's work is based on the assumption that the number of surface particles is constant and hence it ignores the surface (free) energy of the system.

Hui et al. \unskip~\cite{691550:16479843} proposed a new version of Shuttleworth's equation, for which two variables (surface density and temperature) are enough to describe the surface energy, surface tension, and chemical potential.

\clearpage 
    

\bibliographystyle{blank}

\bibliography{\jobname}

\section*{Author biography}

\bioItem[images/bf0f1284-fa36-4678-a9e7-05671376e50c-umeitesm]{Antonio Osamu Katagiri Tanaka}{ .

MNT16

A01212611}
\printBio 

\end{document}
