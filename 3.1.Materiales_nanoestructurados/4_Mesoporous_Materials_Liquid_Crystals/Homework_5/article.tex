\documentclass[11pt]{article}
\makeatletter\if@twocolumn\PassOptionsToPackage{switch}{lineno}\else\fi\makeatother

      \makeatletter
\usepackage{wrapfig}
\newcounter{aubio}

\long\def\bioItem{%
\@ifnextchar[{\@bioItem}{\@@bioItem}}

\long\def\@bioItem[#1]#2#3{
 \stepcounter{aubio}
 \expandafter\gdef\csname authorImage\theaubio\endcsname{#1}
 \expandafter\gdef\csname authorName\theaubio\endcsname{#2}
 \expandafter\gdef\csname authorDetails\theaubio\endcsname{#3}
}

\long\def\@@bioItem#1#2{
 \stepcounter{aubio}
 \expandafter\gdef\csname authorName\theaubio\endcsname{#1}
 \expandafter\gdef\csname authorDetails\theaubio\endcsname{#2}
}

\newcommand{\checkheight}[1]{%
  \par \penalty-100\begingroup%
  \setbox8=\hbox{#1}%
  \setlength{\dimen@}{\ht8}%
  \dimen@ii\pagegoal \advance\dimen@ii-\pagetotal
  \ifdim \dimen@>\dimen@ii
    \break
  \fi\endgroup}

\def\printBio{%
  \@tempcnta=0
   \loop
     \advance \@tempcnta by 1
     \def\aubioCnt{\the\@tempcnta}
     \setlength{\intextsep}{0pt}%
     \setlength{\columnsep}{10pt}%
     \newbox\boxa%
     \setbox\boxa\vbox{\csname authorDetails\aubioCnt\endcsname}
     \expandafter\ifx\csname authorImage\aubioCnt\endcsname\relax%
      \else%
       \checkheight{\includegraphics[height=1.25in,width=1in,keepaspectratio]{\csname authorImage\aubioCnt\endcsname}}
        \begin{wrapfigure}{l}{25mm}
         \includegraphics[height=1.25in,width=1in,keepaspectratio]{\csname authorImage\aubioCnt\endcsname}%height=145pt
        \end{wrapfigure}\par
      \fi
     {\parindent0pt\textbf{\csname authorName\aubioCnt\endcsname}\csname authorDetails\aubioCnt\endcsname \par\bigskip%
     \expandafter\ifx\csname authorImage\aubioCnt\endcsname\relax\else%
      \ifdim\the\ht\boxa < 90pt\vskip\dimexpr(90pt -\the\ht\boxa-1pc)\fi%
     \fi}%for adding additional vskip for avoiding image overlap.
      \ifnum\@tempcnta < \theaubio
   \repeat
   }

\makeatother

      



\usepackage{amsfonts,amssymb,amsbsy,latexsym,amsmath,tabulary,graphicx,times,xcolor}
\usepackage[utf8x]{inputenc}
\usepackage{fancyhdr}
\def\NormalBaseline{\def\baselinestretch{1.1}}
\makeatletter
\def\hlinewd#1{%
  \noalign{\ifnum0=`}\fi\hrule \@height #1%
  \futurelet\reserved@a\@xhline}
\def\tbltoprule{\hlinewd{.8pt}}%\\[-10pt]}
\def\tblbottomrule{\hlinewd{.8pt}}
\def\tblmidrule{\hline\noalign{\vspace*{2pt}}}

\def\@shorttitle{\@empty}
\def\shorttitle#1{\gdef\@shorttitle{#1}}

\fancypagestyle{custom}{
\fancyhf{}
\fancyhead[C]{\@shorttitle}
\fancyhead[R]{\thepage}
\fancyfoot[C]{}
\renewcommand\headrulewidth{0.4pt}
\renewcommand\footrulewidth{0pt}
}
\fancypagestyle{plain}{
\fancyhf{}
\renewcommand\headrulewidth{0.4pt}
}


\makeatother

\usepackage{times}

\usepackage[a4paper,margin=2.5cm,headsep=.7cm,headheight=18pt,top=3cm,footnotesep=1.5\baselineskip]{geometry}
\usepackage{caption}
\captionsetup[figure]{labelfont=bf,labelsep=newline,justification=centerlast,labelfont={small,sc,bf},font=small,aboveskip=.3\baselineskip}

\captionsetup[table]{labelfont=bf,labelsep=newline,justification=centerlast,labelfont={small,sc,bf},font=small,aboveskip=.3\baselineskip}
\linespread{1.5}

\setcounter{totalnumber}{4}
\def\topfraction{0.9}
\def\bottomfraction{0.4}
\def\floatpagefraction{0.8}
\def\textfraction{0.1}
\widowpenalty 10000
\clubpenalty 10000
\makeatletter
\setlength\intextsep   {1.5\baselineskip \@plus 2\p@ \@minus 2\p@}
\makeatother

  
%%%%%%%%%%%%%%%%%%%%%%%%%%%%%%%%%%%%%%%%%%%%%%%%%%%%%%%%%%%%%%%%%%%%%%%%%%
% Following additional macros are required to function some 
% functions which are not available in the class used.
%%%%%%%%%%%%%%%%%%%%%%%%%%%%%%%%%%%%%%%%%%%%%%%%%%%%%%%%%%%%%%%%%%%%%%%%%%
\usepackage{url,multirow,morefloats,floatflt,cancel,tfrupee}
\makeatletter


\AtBeginDocument{\@ifpackageloaded{textcomp}{}{\usepackage{textcomp}}}
\makeatother
\usepackage{colortbl}
\usepackage{xcolor}
\usepackage{pifont}
\usepackage[nointegrals]{wasysym}
\urlstyle{rm}
\makeatletter

%%%For Table column width calculation.
\def\mcWidth#1{\csname TY@F#1\endcsname+\tabcolsep}

%%Hacking center and right align for table
\def\cAlignHack{\rightskip\@flushglue\leftskip\@flushglue\parindent\z@\parfillskip\z@skip}
\def\rAlignHack{\rightskip\z@skip\leftskip\@flushglue \parindent\z@\parfillskip\z@skip}

%Etal definition in references
\@ifundefined{etal}{\def\etal{\textit{et~al}}}{}


%\if@twocolumn\usepackage{dblfloatfix}\fi
\usepackage{ifxetex}
\ifxetex\else\if@twocolumn\@ifpackageloaded{stfloats}{}{\usepackage{dblfloatfix}}\fi\fi

\AtBeginDocument{
\expandafter\ifx\csname eqalign\endcsname\relax
\def\eqalign#1{\null\vcenter{\def\\{\cr}\openup\jot\m@th
  \ialign{\strut$\displaystyle{##}$\hfil&$\displaystyle{{}##}$\hfil
      \crcr#1\crcr}}\,}
\fi
}

%For fixing hardfail when unicode letters appear inside table with endfloat
\AtBeginDocument{%
  \@ifpackageloaded{endfloat}%
   {\renewcommand\efloat@iwrite[1]{\immediate\expandafter\protected@write\csname efloat@post#1\endcsname{}}}{\newif\ifefloat@tables}%
}%

\def\BreakURLText#1{\@tfor\brk@tempa:=#1\do{\brk@tempa\hskip0pt}}
\let\lt=<
\let\gt=>
\def\processVert{\ifmmode|\else\textbar\fi}
\let\processvert\processVert

\@ifundefined{subparagraph}{
\def\subparagraph{\@startsection{paragraph}{5}{2\parindent}{0ex plus 0.1ex minus 0.1ex}%
{0ex}{\normalfont\small\itshape}}%
}{}

% These are now gobbled, so won't appear in the PDF.
\newcommand\role[1]{\unskip}
\newcommand\aucollab[1]{\unskip}
  
\@ifundefined{tsGraphicsScaleX}{\gdef\tsGraphicsScaleX{1}}{}
\@ifundefined{tsGraphicsScaleY}{\gdef\tsGraphicsScaleY{.9}}{}
% To automatically resize figures to fit inside the text area
\def\checkGraphicsWidth{\ifdim\Gin@nat@width>\linewidth
	\tsGraphicsScaleX\linewidth\else\Gin@nat@width\fi}

\def\checkGraphicsHeight{\ifdim\Gin@nat@height>.9\textheight
	\tsGraphicsScaleY\textheight\else\Gin@nat@height\fi}

\def\fixFloatSize#1{}%\@ifundefined{processdelayedfloats}{\setbox0=\hbox{\includegraphics{#1}}\ifnum\wd0<\columnwidth\relax\renewenvironment{figure*}{\begin{figure}}{\end{figure}}\fi}{}}
\let\ts@includegraphics\includegraphics

\def\inlinegraphic[#1]#2{{\edef\@tempa{#1}\edef\baseline@shift{\ifx\@tempa\@empty0\else#1\fi}\edef\tempZ{\the\numexpr(\numexpr(\baseline@shift*\f@size/100))}\protect\raisebox{\tempZ pt}{\ts@includegraphics{#2}}}}

%\renewcommand{\includegraphics}[1]{\ts@includegraphics[width=\checkGraphicsWidth]{#1}}
\AtBeginDocument{\def\includegraphics{\@ifnextchar[{\ts@includegraphics}{\ts@includegraphics[width=\checkGraphicsWidth,height=\checkGraphicsHeight,keepaspectratio]}}}

\DeclareMathAlphabet{\mathpzc}{OT1}{pzc}{m}{it}

\def\URL#1#2{\@ifundefined{href}{#2}{\href{#1}{#2}}}

%%For url break
\def\UrlOrds{\do\*\do\-\do\~\do\'\do\"\do\-}%
\g@addto@macro{\UrlBreaks}{\UrlOrds}



\edef\fntEncoding{\f@encoding}
\def\EUoneEnc{EU1}
\makeatother
\def\floatpagefraction{0.8} 
\def\dblfloatpagefraction{0.8}
\def\style#1#2{#2}
\def\xxxguillemotleft{\fontencoding{T1}\selectfont\guillemotleft}
\def\xxxguillemotright{\fontencoding{T1}\selectfont\guillemotright}

\newif\ifmultipleabstract\multipleabstractfalse%
\newenvironment{typesetAbstractGroup}{}{}%

%%%%%%%%%%%%%%%%%%%%%%%%%%%%%%%%%%%%%%%%%%%%%%%%%%%%%%%%%%%%%%%%%%%%%%%%%%

\usepackage{natbib}




\usepackage{titlesec}
\usepackage[T1]{fontenc}
\setcounter{secnumdepth}{5}
 
\titleformat{\section}[hang]{\NormalBaseline\filright\large\bfseries}
{\large\thesection}
{10pt}
{}
[]
\titleformat{\subsection}[hang]{\NormalBaseline\filright\bfseries}
{\thesubsection}
{10pt}
{}
[]
\titleformat{\subsubsection}[hang]{\NormalBaseline\filright\bfseries\itshape}
{\upshape\thesubsubsection}
{10pt}
{}
[]
\titleformat{\paragraph}[runin]{\NormalBaseline\filright\bfseries}
{\theparagraph}
{10pt}
{}
[]
\titleformat{\subparagraph}[runin]{\NormalBaseline\filright\bfseries\itshape}
{\thesubparagraph}
{10pt}
{}
[]

\titlespacing{\section}{0pt}{1.5\baselineskip}{.2\baselineskip}  
\titlespacing{\subsection}{0pt}{1.5\baselineskip}{.2\baselineskip}  
\titlespacing{\subsubsection}{0pt}{1.5\baselineskip}{.2\baselineskip}  
\titlespacing{\paragraph}{0pt}{.5\baselineskip}{10pt}  
\titlespacing{\subparagraph}{0pt}{.5\baselineskip}{10pt}  
  

  




%%%%%%%%%%%%%%%%%%%%%%%%%%%%%%%%%%%%%%%%%%
% Feature enabled:
%secnum: unnumbered
%linespacing: 1.25
%%%%%%%%%%%%%%%%%%%%%%%%%%%%%%%%%%%%%%%%%%

\setcounter{secnumdepth}{0}

\linespread{1.25}

\begin{document}



\renewcommand*\rmdefault{bch}\normalfont\upshape

\shorttitle{Sol-Gel Method}

\date{}  

  
\title{\NormalBaseline\raggedright\bfseries Homework 5 {\textemdash} Mesoporous Materials, templated by amphiphilic molecules}
  \let\origthanks\thanks
\renewcommand\thanks[1]{\begingroup\let\rlap\relax\origthanks{#1}\endgroup}
\author{\hskip2pc\parbox{.95\textwidth}{\bfseries\large Chinmay Pramodkumar Tiwari\textsuperscript{1}\thanks{E-mail: A00827392@itesm.mx}, 
        Antonio Osamu Katagiri Tanaka\textsuperscript{1}\thanks{E-mail: A01212611@itesm.mx}
      \\[3pt] 
    % Address
    \normalfont\itshape\NormalBaseline \textsuperscript{\upshape 1} 
    ITESM\unskip, \normalfont\itshape\NormalBaseline Av. Eugenio Garza Sada 2501 Sur\unskip, N.L.\unskip, Monterrey\unskip, Mexico}}
    
    
\maketitle 
\pagestyle{custom}

    
\section{Paper Review {\textemdash} Study of the Pluronic\ensuremath{-}Silica Interaction in Synthesis of Mesoporous Silica under Mild Acidic Conditions\unskip~\protect\cite{691550:16985251}.}
Sundblom et al. investigated the interaction between silica and poly(ethylene oxide) (PEO), using Pluronic and tetraethylorthosilicate (TEOS) as PEO and silica sources respectively. The authors state that the silica source has a strong influence on the rate of material formation and therefore on the structure of the final material, however this is not assessed in this paper. In summary, TEOS was first hydrolyzed followed by subsequently Pluronic additions at varying times after hydrolysis. UV-Vis spectroscopy was implemented to characterize the turbidity of the solutions and dynamic light scattering (DLS) to measure the average aggregate size.

During characterization, the authors discovered that the size of the aggregates increases by the addition of the silica, by increasing temperature, and as the time between hydrolysis of TEOS and addition of Pluronic increased. Both the reaction temperature and the time between hydrolysis of TEOS and addition of Pluronic define the porosity of the material and the degree of order.

As the temperature increased the solubility of both poly(propylene oxide) PPO and PEO blocks decreased. As PPO is less soluble in water than PEO, PPO initiates the formation of aggregates in the solution. Further increase in temperature lead to the formation of ordinary spherical micelles. The increase in temperature causes a reduction of the amount of water in the PEO corona micelle, following a reduction of the headgroup area. This translates in an increase in the critical packing parameter concept (CPP) value. Shrinkage of the headgroup leads to lower temperature for the sphere-to-rod transition and an increased attraction between micelles. The temperature for the sphere-to-rod transition decreases with an increase of the degree of silica condensation.

The influence of the degree of polymerization can be described in changes in Gibbs free energy. In one hand, the change in enthalpy is to be of minor importance as the number and the strength of the hydrogen bonds formed and broken are relatively equal. On the other hand, the higher the silica molecular weight, the more water molecules are released, generating a greater change in entropy. Hence, the increased attraction between the silica and the PEO as the degree of silica polymerization increases. The order improves as the hydrolysis time increased. To obtain well-ordered materials, the micellar aggregates growth and phase separation processes must be fast relative to the condensation of the silica in the corona of the micelles.
    
\section{Class Assignment {\textemdash} Let us Talk about Laboratory Experiences to come (TEOS, Pluronic, SDS, and CTAC).}
 Surfactants are frequently used in the synthesis of mesoporous silica nanoparticles (MSNs) due to their tendency to form micelles, therefore silica sources tend to aggregate and form silica networks around these micelles. Ionic surfactants define MSN templating through electrostatic interactions between the surfactant micelles and the silicate source. For instance, a typical cationic surfactant and anionic silicate source is hexadecyltrimethylammonium chloride (CTAC) and tetraetil ortosilicato (TEOS)\unskip~\cite{691550:16986475}.

Roughly, pore size control is limited to the used template, however varying chain lengths in CTAC can impact the final pore size of the MSN\unskip~\cite{691550:16987569}.

Due to the anionic nature of silicate precursors, an intermediate (co-silica) source with cationic behavior is needed in order to adapt the anionic surfactant templating method\unskip~\cite{691550:16987570}. Che et al. an achieved a charge neutralization effect with the addition of sodium dodecyl sulfate (SDS). SDS is a negative anionic that interacts with the positively charged surfactant, in the addition of TEOS.

The most used non.ionic/amphiphilic templates are tri-block polymers, such as PEO-PPO-PEO known as Pluronic. Due to the hydrophilic nature of PEO and the hydrophobic nature of PPO, Pluronics behave as amphiphilic surfactants, with varying strength in relation to the polymer chain length and ratio. One of the earliest MSN involving non-ionic surfactants was done by\unskip~\cite{691550:16987571}. As reported by Zhao et al., the formation of uniform MSNs is dependent to all reaction parameters, including temperature, pH, and the template concentration. When using PEO-PPO-PEO as the template, greater PEO content results in cubic structures. The presence of a co-solvent (ethanol) also favors the conversion of hexagonal to cubic morphology\unskip~\cite{691550:16987628}.



\subsection{Proposed experiments}As per revisited literature, the proposed experiment shall include two sample sets. One set shall include ionic surfactants (CTAC and SDS). The other sample set shall assess non-ionic surfactants (Pluronic). In both sample sets, TEOS shall be used as the silica source. Different template concentrations, reaction temperatures and pH shall be investigated. Spectroscopy characterization techniques could be implemented to study the MSN morphology.

\clearpage 


    

\bibliographystyle{blank}

\bibliography{\jobname}

\section*{Author biography}

\bioItem[images/6df1cd17-afe7-41d0-a87c-70c81d58cbfc-uchinmay]{Chinmay Pramodkumar Tiwari}{ .

MNT16

A00827392}

\smallskip\noindent 

\bioItem[images/bf0f1284-fa36-4678-a9e7-05671376e50c-umeitesm]{Antonio Osamu Katagiri Tanaka}{ .

MNT16

A01212611}
\printBio 

\end{document}
