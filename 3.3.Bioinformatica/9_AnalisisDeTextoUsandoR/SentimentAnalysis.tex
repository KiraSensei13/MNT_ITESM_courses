% Options for packages loaded elsewhere
\PassOptionsToPackage{unicode}{hyperref}
\PassOptionsToPackage{hyphens}{url}
%
\documentclass[
  a4paper]{article}
\usepackage{lmodern}
\usepackage{amssymb,amsmath}
\usepackage{ifxetex,ifluatex}
\ifnum 0\ifxetex 1\fi\ifluatex 1\fi=0 % if pdftex
  \usepackage[T1]{fontenc}
  \usepackage[utf8]{inputenc}
  \usepackage{textcomp} % provide euro and other symbols
\else % if luatex or xetex
  \usepackage{unicode-math}
  \defaultfontfeatures{Scale=MatchLowercase}
  \defaultfontfeatures[\rmfamily]{Ligatures=TeX,Scale=1}
\fi
% Use upquote if available, for straight quotes in verbatim environments
\IfFileExists{upquote.sty}{\usepackage{upquote}}{}
\IfFileExists{microtype.sty}{% use microtype if available
  \usepackage[]{microtype}
  \UseMicrotypeSet[protrusion]{basicmath} % disable protrusion for tt fonts
}{}
\makeatletter
\@ifundefined{KOMAClassName}{% if non-KOMA class
  \IfFileExists{parskip.sty}{%
    \usepackage{parskip}
  }{% else
    \setlength{\parindent}{0pt}
    \setlength{\parskip}{6pt plus 2pt minus 1pt}}
}{% if KOMA class
  \KOMAoptions{parskip=half}}
\makeatother
\usepackage{xcolor}
\IfFileExists{xurl.sty}{\usepackage{xurl}}{} % add URL line breaks if available
\IfFileExists{bookmark.sty}{\usepackage{bookmark}}{\usepackage{hyperref}}
\hypersetup{
  pdftitle={Sentiment Analysis},
  pdfauthor={Antonio Osamu Katagiri Tanaka - A01212611@itesm.mx},
  hidelinks,
  pdfcreator={LaTeX via pandoc}}
\urlstyle{same} % disable monospaced font for URLs
\usepackage[margin=1.75cm]{geometry}
\usepackage{color}
\usepackage{fancyvrb}
\newcommand{\VerbBar}{|}
\newcommand{\VERB}{\Verb[commandchars=\\\{\}]}
\DefineVerbatimEnvironment{Highlighting}{Verbatim}{commandchars=\\\{\}}
% Add ',fontsize=\small' for more characters per line
\usepackage{framed}
\definecolor{shadecolor}{RGB}{248,248,248}
\newenvironment{Shaded}{\begin{snugshade}}{\end{snugshade}}
\newcommand{\AlertTok}[1]{\textcolor[rgb]{0.94,0.16,0.16}{#1}}
\newcommand{\AnnotationTok}[1]{\textcolor[rgb]{0.56,0.35,0.01}{\textbf{\textit{#1}}}}
\newcommand{\AttributeTok}[1]{\textcolor[rgb]{0.77,0.63,0.00}{#1}}
\newcommand{\BaseNTok}[1]{\textcolor[rgb]{0.00,0.00,0.81}{#1}}
\newcommand{\BuiltInTok}[1]{#1}
\newcommand{\CharTok}[1]{\textcolor[rgb]{0.31,0.60,0.02}{#1}}
\newcommand{\CommentTok}[1]{\textcolor[rgb]{0.56,0.35,0.01}{\textit{#1}}}
\newcommand{\CommentVarTok}[1]{\textcolor[rgb]{0.56,0.35,0.01}{\textbf{\textit{#1}}}}
\newcommand{\ConstantTok}[1]{\textcolor[rgb]{0.00,0.00,0.00}{#1}}
\newcommand{\ControlFlowTok}[1]{\textcolor[rgb]{0.13,0.29,0.53}{\textbf{#1}}}
\newcommand{\DataTypeTok}[1]{\textcolor[rgb]{0.13,0.29,0.53}{#1}}
\newcommand{\DecValTok}[1]{\textcolor[rgb]{0.00,0.00,0.81}{#1}}
\newcommand{\DocumentationTok}[1]{\textcolor[rgb]{0.56,0.35,0.01}{\textbf{\textit{#1}}}}
\newcommand{\ErrorTok}[1]{\textcolor[rgb]{0.64,0.00,0.00}{\textbf{#1}}}
\newcommand{\ExtensionTok}[1]{#1}
\newcommand{\FloatTok}[1]{\textcolor[rgb]{0.00,0.00,0.81}{#1}}
\newcommand{\FunctionTok}[1]{\textcolor[rgb]{0.00,0.00,0.00}{#1}}
\newcommand{\ImportTok}[1]{#1}
\newcommand{\InformationTok}[1]{\textcolor[rgb]{0.56,0.35,0.01}{\textbf{\textit{#1}}}}
\newcommand{\KeywordTok}[1]{\textcolor[rgb]{0.13,0.29,0.53}{\textbf{#1}}}
\newcommand{\NormalTok}[1]{#1}
\newcommand{\OperatorTok}[1]{\textcolor[rgb]{0.81,0.36,0.00}{\textbf{#1}}}
\newcommand{\OtherTok}[1]{\textcolor[rgb]{0.56,0.35,0.01}{#1}}
\newcommand{\PreprocessorTok}[1]{\textcolor[rgb]{0.56,0.35,0.01}{\textit{#1}}}
\newcommand{\RegionMarkerTok}[1]{#1}
\newcommand{\SpecialCharTok}[1]{\textcolor[rgb]{0.00,0.00,0.00}{#1}}
\newcommand{\SpecialStringTok}[1]{\textcolor[rgb]{0.31,0.60,0.02}{#1}}
\newcommand{\StringTok}[1]{\textcolor[rgb]{0.31,0.60,0.02}{#1}}
\newcommand{\VariableTok}[1]{\textcolor[rgb]{0.00,0.00,0.00}{#1}}
\newcommand{\VerbatimStringTok}[1]{\textcolor[rgb]{0.31,0.60,0.02}{#1}}
\newcommand{\WarningTok}[1]{\textcolor[rgb]{0.56,0.35,0.01}{\textbf{\textit{#1}}}}
\usepackage{graphicx,grffile}
\makeatletter
\def\maxwidth{\ifdim\Gin@nat@width>\linewidth\linewidth\else\Gin@nat@width\fi}
\def\maxheight{\ifdim\Gin@nat@height>\textheight\textheight\else\Gin@nat@height\fi}
\makeatother
% Scale images if necessary, so that they will not overflow the page
% margins by default, and it is still possible to overwrite the defaults
% using explicit options in \includegraphics[width, height, ...]{}
\setkeys{Gin}{width=\maxwidth,height=\maxheight,keepaspectratio}
% Set default figure placement to htbp
\makeatletter
\def\fps@figure{htbp}
\makeatother
\setlength{\emergencystretch}{3em} % prevent overfull lines
\providecommand{\tightlist}{%
  \setlength{\itemsep}{0pt}\setlength{\parskip}{0pt}}
\setcounter{secnumdepth}{-\maxdimen} % remove section numbering

\title{Sentiment Analysis}
\author{Antonio Osamu Katagiri Tanaka -
\href{mailto:A01212611@itesm.mx}{\nolinkurl{A01212611@itesm.mx}}}
\date{May 01, 2020}

\begin{document}
\maketitle

\hypertarget{part-1-paper-mining}{%
\section{Part 1: paper mining}\label{part-1-paper-mining}}

\hypertarget{load-libraries-and-set-custom-settings}{%
\subsection{Load libraries and set custom
settings}\label{load-libraries-and-set-custom-settings}}

\begin{Shaded}
\begin{Highlighting}[]
\CommentTok{# Clear all objects (from the workspace)}
\KeywordTok{rm}\NormalTok{(}\DataTypeTok{list =} \KeywordTok{ls}\NormalTok{())}

\CommentTok{# Strings are not factors}
\KeywordTok{options}\NormalTok{(}\DataTypeTok{stringsAsFactors =}\NormalTok{ F)}

\CommentTok{# Install and load libraries}
\KeywordTok{library}\NormalTok{(RISmed) }\CommentTok{#PUBmed}
\KeywordTok{library}\NormalTok{(tm)}
\end{Highlighting}
\end{Shaded}

\begin{verbatim}
## Loading required package: NLP
\end{verbatim}

\hypertarget{define-the-requested-query-and-seek}{%
\subsection{Define the requested query and
seek}\label{define-the-requested-query-and-seek}}

\begin{Shaded}
\begin{Highlighting}[]
\NormalTok{query_colon <-}
\StringTok{    "}\CharTok{\textbackslash{}"}\StringTok{electrospinning}\CharTok{\textbackslash{}"}\StringTok{[TIAB] AND (}\CharTok{\textbackslash{}"}\StringTok{NFES}\CharTok{\textbackslash{}"}\StringTok{[TIAB] OR (}\CharTok{\textbackslash{}"}\StringTok{near}\CharTok{\textbackslash{}"}\StringTok{[TIAB] AND }\CharTok{\textbackslash{}"}\StringTok{field}\CharTok{\textbackslash{}"}\StringTok{[TIAB]))"}
\NormalTok{search_query <-}\StringTok{ }\KeywordTok{EUtilsSummary}\NormalTok{(query_colon)}

\CommentTok{# Let's take a look}
\KeywordTok{summary}\NormalTok{(search_query)}
\end{Highlighting}
\end{Shaded}

\begin{verbatim}
## Query:
## "electrospinning"[TIAB] AND ("NFES"[TIAB] OR ("near"[TIAB] AND "field"[TIAB])) 
## 
## Result count:  66
\end{verbatim}

\hypertarget{fetch-the-data-as-dataframes}{%
\subsection{Fetch the data as
dataframes}\label{fetch-the-data-as-dataframes}}

\begin{Shaded}
\begin{Highlighting}[]
\NormalTok{records <-}\StringTok{ }\KeywordTok{EUtilsGet}\NormalTok{(search_query)}
\NormalTok{pubmed_data <-}
\StringTok{    }\KeywordTok{data.frame}\NormalTok{(}
        \StringTok{'Title'}\NormalTok{ =}\StringTok{ }\KeywordTok{ArticleTitle}\NormalTok{(records),}
        \StringTok{'Abstract'}\NormalTok{ =}\StringTok{ }\KeywordTok{AbstractText}\NormalTok{(records),}
        \StringTok{'PID'}\NormalTok{ =}\StringTok{ }\KeywordTok{ArticleId}\NormalTok{(records)}
\NormalTok{    )}

\CommentTok{# Let's take a look to the 1st search}
\NormalTok{pubmed_data[}\DecValTok{1}\NormalTok{, ]}
\end{Highlighting}
\end{Shaded}

\begin{verbatim}
##                                                                                                                                       Title
## 1 Fiber Lithography: A Facile Lithography Platform Based on Electromagnetic Phase Modulation Using a Highly Birefringent Electrospun Fiber.
##                                                                                                                                                                                                                                                                                                                                                                                                                                                                                                                                                                                                                                                                                                                                                                                                                                                                                                                                                                                                                                                                                                                                                                                                                                                                                                                                                                                                                                                                                                                                                                                                                                                                                                                                                                                               Abstract
## 1 Lithography plays a key role in advancing manufacturing as well as the semiconductor industry. However, the currently available state-of-the-art lithography methods still require access to expensive tools and facilities. Herein, we suggest a novel lithography method based on electromagnetic phase modulation of ultraviolet using a highly birefringent electrospun fiber to overcome such limitations. The optical birefringent effect, by which the phase of incident ultraviolet electromagnetic fields is retarded when passing through optically anisotropic media, is combined with semicrystalline poly(ethylene oxide) (PEO)-poly(ethylene glycol) (PEG) polymeric microfibers patterned in a programmable form using near-field electrospinning. By positioning the mask between two linear polarizers that are perpendicular to each other, only the UV waves that are passing through the fibers can be selectively utilized to exhibit lithographic property. Therefore, the UV intensity on the photoresist (PR) surface follows the shape of the fiber pattern, enabling precisely controlled patterning of the photoresist. Zero- to two-dimensional key features of lithography are achieved, including straight, curved, array, and isolated patterns. Facile optical alignments without using dedicated alignment marks are successfully demonstrated, as well as various applications including micro- to macroscale serpentine, tree, and antenna circuit patterns on a flexible substrate. The presented approach, packed in a table-top scale, is expected to provide a practical and affordable lithography solution by leveraging the direct-writing capability and tunable optical functionality of polymers, scalability, and the simple optical alignment method.
##        PID
## 1 32297731
\end{verbatim}

\hypertarget{process-the-data}{%
\subsection{Process the data}\label{process-the-data}}

\begin{Shaded}
\begin{Highlighting}[]
\CommentTok{# Remove characters : , ; [ ] ( ) from titles and abstracts}
\NormalTok{pubmed_data}\OperatorTok{$}\NormalTok{Title <-}
\StringTok{    }\KeywordTok{gsub}\NormalTok{(}\DataTypeTok{pattern =} \StringTok{"}\CharTok{\textbackslash{}\textbackslash{}}\StringTok{.|:|,|;|}\CharTok{\textbackslash{}\textbackslash{}}\StringTok{[|}\CharTok{\textbackslash{}\textbackslash{}}\StringTok{]|}\CharTok{\textbackslash{}\textbackslash{}}\StringTok{(|}\CharTok{\textbackslash{}\textbackslash{}}\StringTok{)|-"}\NormalTok{,}
         \DataTypeTok{replacement =} \StringTok{""}\NormalTok{,}
\NormalTok{         pubmed_data}\OperatorTok{$}\NormalTok{Title)}
\NormalTok{pubmed_data}\OperatorTok{$}\NormalTok{Abstract <-}
\StringTok{    }\KeywordTok{gsub}\NormalTok{(}\DataTypeTok{pattern =} \StringTok{"}\CharTok{\textbackslash{}\textbackslash{}}\StringTok{.|:|,|;|}\CharTok{\textbackslash{}\textbackslash{}}\StringTok{[|}\CharTok{\textbackslash{}\textbackslash{}}\StringTok{]|}\CharTok{\textbackslash{}\textbackslash{}}\StringTok{(|}\CharTok{\textbackslash{}\textbackslash{}}\StringTok{)|-"}\NormalTok{,}
         \DataTypeTok{replacement =} \StringTok{""}\NormalTok{,}
\NormalTok{         pubmed_data}\OperatorTok{$}\NormalTok{Abstract)}

\CommentTok{# Remove upper case in titles and abstracts}
\NormalTok{pubmed_data}\OperatorTok{$}\NormalTok{Title <-}\StringTok{ }\KeywordTok{tolower}\NormalTok{(pubmed_data}\OperatorTok{$}\NormalTok{Title)}
\NormalTok{pubmed_data}\OperatorTok{$}\NormalTok{Abstract <-}\StringTok{ }\KeywordTok{tolower}\NormalTok{(pubmed_data}\OperatorTok{$}\NormalTok{Abstract)}

\CommentTok{# Let's take a look to the 1st search}
\NormalTok{pubmed_data[}\DecValTok{1}\NormalTok{, ]}
\end{Highlighting}
\end{Shaded}

\begin{verbatim}
##                                                                                                                                     Title
## 1 fiber lithography a facile lithography platform based on electromagnetic phase modulation using a highly birefringent electrospun fiber
##                                                                                                                                                                                                                                                                                                                                                                                                                                                                                                                                                                                                                                                                                                                                                                                                                                                                                                                                                                                                                                                                                                                                                                                                                                                                                                                                                                                                                                                                                                                                                                                                                                                                                                                                                Abstract
## 1 lithography plays a key role in advancing manufacturing as well as the semiconductor industry however the currently available stateoftheart lithography methods still require access to expensive tools and facilities herein we suggest a novel lithography method based on electromagnetic phase modulation of ultraviolet using a highly birefringent electrospun fiber to overcome such limitations the optical birefringent effect by which the phase of incident ultraviolet electromagnetic fields is retarded when passing through optically anisotropic media is combined with semicrystalline polyethylene oxide peopolyethylene glycol peg polymeric microfibers patterned in a programmable form using nearfield electrospinning by positioning the mask between two linear polarizers that are perpendicular to each other only the uv waves that are passing through the fibers can be selectively utilized to exhibit lithographic property therefore the uv intensity on the photoresist pr surface follows the shape of the fiber pattern enabling precisely controlled patterning of the photoresist zero to twodimensional key features of lithography are achieved including straight curved array and isolated patterns facile optical alignments without using dedicated alignment marks are successfully demonstrated as well as various applications including micro to macroscale serpentine tree and antenna circuit patterns on a flexible substrate the presented approach packed in a tabletop scale is expected to provide a practical and affordable lithography solution by leveraging the directwriting capability and tunable optical functionality of polymers scalability and the simple optical alignment method
##        PID
## 1 32297731
\end{verbatim}

\begin{Shaded}
\begin{Highlighting}[]
\CommentTok{# Are there empty abstracts?}
\KeywordTok{which}\NormalTok{(pubmed_data}\OperatorTok{$}\NormalTok{Abstract }\OperatorTok{==}\StringTok{ ""}\NormalTok{)}
\end{Highlighting}
\end{Shaded}

\begin{verbatim}
## [1]  6  9 11 14 27 35 49
\end{verbatim}

\begin{Shaded}
\begin{Highlighting}[]
\CommentTok{# Fetch the words within all abstracts in a dataframe.}

\CommentTok{# data frame para guardar las palabras}
\NormalTok{word_list <-}\StringTok{ }\KeywordTok{c}\NormalTok{()}

\CommentTok{#Ciclo para todos los abstracts}
\ControlFlowTok{for}\NormalTok{ (i }\ControlFlowTok{in} \DecValTok{1}\OperatorTok{:}\KeywordTok{length}\NormalTok{(pubmed_data}\OperatorTok{$}\NormalTok{Abstract)) \{}
    \CommentTok{#Obtener las palabras como vector en lugar de lista}
\NormalTok{    titlePabstract <-}\StringTok{ }\KeywordTok{paste}\NormalTok{(pubmed_data}\OperatorTok{$}\NormalTok{Title[i], pubmed_data}\OperatorTok{$}\NormalTok{Abstract[i], }\DataTypeTok{sep =} \StringTok{" "}\NormalTok{)}
\NormalTok{    aux_word <-}\StringTok{ }\KeywordTok{unlist}\NormalTok{(}\KeywordTok{strsplit}\NormalTok{(titlePabstract, }\StringTok{" "}\NormalTok{))}
    \CommentTok{#aux_word <- unlist(strsplit(pubmed_data$Abstract[i], " "))}
    
    \CommentTok{#Si el abstract tiene palabras}
    \ControlFlowTok{if}\NormalTok{ (}\KeywordTok{length}\NormalTok{(aux_word) }\OperatorTok{>}\StringTok{ }\DecValTok{0}\NormalTok{) \{}
        \CommentTok{#Se juntan las palabras y el PUBMED ID}
\NormalTok{        aux_list <-}\StringTok{ }\KeywordTok{cbind}\NormalTok{(pubmed_data}\OperatorTok{$}\NormalTok{PID[i], aux_word)}
        
        \CommentTok{#Se pega este data frame auxiliar al que guarda todo}
\NormalTok{        word_list <-}\StringTok{ }\KeywordTok{rbind}\NormalTok{(word_list, aux_list)}
\NormalTok{    \}}
\NormalTok{\}}
\KeywordTok{colnames}\NormalTok{(word_list) <-}\StringTok{ }\KeywordTok{c}\NormalTok{(}\StringTok{"PID"}\NormalTok{, }\StringTok{"Word"}\NormalTok{)}

\CommentTok{# Let's take a look}
\KeywordTok{dim}\NormalTok{(word_list)}
\end{Highlighting}
\end{Shaded}

\begin{verbatim}
## [1] 11544     2
\end{verbatim}

\begin{Shaded}
\begin{Highlighting}[]
\CommentTok{# Let's take a look}
\KeywordTok{head}\NormalTok{(word_list)}
\end{Highlighting}
\end{Shaded}

\begin{verbatim}
##      PID        Word         
## [1,] "32297731" "fiber"      
## [2,] "32297731" "lithography"
## [3,] "32297731" "a"          
## [4,] "32297731" "facile"     
## [5,] "32297731" "lithography"
## [6,] "32297731" "platform"
\end{verbatim}

\begin{Shaded}
\begin{Highlighting}[]
\CommentTok{# Remove stopwords with tm}

\CommentTok{# Fetch the English stop_words from tm DB}
\NormalTok{stop_words <-}\StringTok{ }\KeywordTok{stopwords}\NormalTok{(}\DataTypeTok{kind =} \StringTok{"en"}\NormalTok{)}
\KeywordTok{head}\NormalTok{(stop_words)}
\end{Highlighting}
\end{Shaded}

\begin{verbatim}
## [1] "i"      "me"     "my"     "myself" "we"     "our"
\end{verbatim}

\begin{Shaded}
\begin{Highlighting}[]
\CommentTok{# Use the indexes to remove stopwords}
\NormalTok{index_stop_word <-}\StringTok{ }\KeywordTok{which}\NormalTok{(word_list[, }\DecValTok{2}\NormalTok{] }\OperatorTok\StringTok{ }\NormalTok{stop_words)}

\CommentTok{# Let's take a look}
\KeywordTok{dim}\NormalTok{(word_list)}
\end{Highlighting}
\end{Shaded}

\begin{verbatim}
## [1] 11544     2
\end{verbatim}

\begin{Shaded}
\begin{Highlighting}[]
\NormalTok{word_list <-}\StringTok{ }\NormalTok{word_list[}\OperatorTok{-}\NormalTok{index_stop_word, ]}

\CommentTok{# Let's take a look}
\KeywordTok{dim}\NormalTok{(word_list)}
\end{Highlighting}
\end{Shaded}

\begin{verbatim}
## [1] 7605    2
\end{verbatim}

\begin{Shaded}
\begin{Highlighting}[]
\CommentTok{# Show the 10 most popular words}
\KeywordTok{sort}\NormalTok{(}\KeywordTok{table}\NormalTok{(word_list[,}\DecValTok{2}\NormalTok{]), }\DataTypeTok{decreasing=}\NormalTok{T)[}\DecValTok{1}\OperatorTok{:}\DecValTok{10}\NormalTok{]}
\end{Highlighting}
\end{Shaded}

\begin{verbatim}
## 
## electrospinning          fibers      nanofibers       nearfield             can 
##             141              88              68              67              58 
##         polymer              3d     electrospun           fiber           using 
##              51              49              48              46              44
\end{verbatim}

\begin{Shaded}
\begin{Highlighting}[]
\CommentTok{# Remove duplicated words within each abstract}

\CommentTok{# Identify each word's abstract origin}
\NormalTok{word_df <-}\StringTok{ }\KeywordTok{data.frame}\NormalTok{(}\DataTypeTok{PID=}\KeywordTok{as.numeric}\NormalTok{(word_list[,}\DecValTok{1}\NormalTok{]), }\DataTypeTok{Word=}\NormalTok{word_list[,}\DecValTok{2}\NormalTok{],}
\DataTypeTok{PIDWord=}\KeywordTok{as.character}\NormalTok{(}\KeywordTok{apply}\NormalTok{(word_list, }\DecValTok{1}\NormalTok{, paste, }\DataTypeTok{collapse=}\StringTok{"_"}\NormalTok{)))}

\CommentTok{# Remove duplicates}
\NormalTok{dup_index <-}\StringTok{ }\KeywordTok{duplicated}\NormalTok{(word_df}\OperatorTok{$}\NormalTok{PIDWord)}
\KeywordTok{dim}\NormalTok{(word_df) }\CommentTok{# Let's take a look}
\end{Highlighting}
\end{Shaded}

\begin{verbatim}
## [1] 7605    3
\end{verbatim}

\begin{Shaded}
\begin{Highlighting}[]
\NormalTok{word_df <-}\StringTok{ }\NormalTok{word_df[}\OperatorTok{-}\KeywordTok{which}\NormalTok{(dup_index),]}

\CommentTok{# Let's take a look}
\KeywordTok{dim}\NormalTok{(word_df)}
\end{Highlighting}
\end{Shaded}

\begin{verbatim}
## [1] 5678    3
\end{verbatim}

\begin{Shaded}
\begin{Highlighting}[]
\CommentTok{# Show the 50 most popular words (no duplicates)}
\KeywordTok{sort}\NormalTok{(}\KeywordTok{table}\NormalTok{(word_df}\OperatorTok{$}\NormalTok{Word), }\DataTypeTok{decreasing=}\NormalTok{T)[}\DecValTok{1}\OperatorTok{:}\DecValTok{50}\NormalTok{]}
\end{Highlighting}
\end{Shaded}

\begin{verbatim}
## 
## electrospinning       nearfield             can          fibers    applications 
##              56              40              32              31              28 
##           using     electrospun           field      nanofibers         polymer 
##              26              25              25              23              23 
##           fiber            nfes         process          method            near 
##              20              20              20              18              18 
##       technique       potential       substrate          tissue            used 
##              18              17              17              17              17 
##     engineering     fabrication            also           cells         however 
##              16              16              15              15              15 
##       materials         voltage      fabricated        solution           study 
##              15              15              14              14              14 
##           based            cell         control       different            high 
##              13              13              13              13              13 
##             new           paper             use         applied     development 
##              13              13              13              12              12 
##          direct             low              nm           oxide        patterns 
##              12              12              12              12              12 
##      properties         results      structures             via              3d 
##              12              12              12              12              11
\end{verbatim}

\hypertarget{lets-take-a-look-to-specific-words}{%
\subsection{Let's take a look to specific
words}\label{lets-take-a-look-to-specific-words}}

\begin{Shaded}
\begin{Highlighting}[]
\NormalTok{word_df <-}\StringTok{ }\NormalTok{word_df[}\KeywordTok{order}\NormalTok{(word_df}\OperatorTok{$}\NormalTok{PID, }\DataTypeTok{decreasing=}\NormalTok{T),]}
\NormalTok{index_genes <-}\StringTok{ }\KeywordTok{which}\NormalTok{(word_df}\OperatorTok{$}\NormalTok{Word }\OperatorTok\StringTok{ }\KeywordTok{c}\NormalTok{(}\StringTok{"pyrolysis"}\NormalTok{, }\StringTok{"carbon"}\NormalTok{, }\StringTok{"conductivity"}\NormalTok{))}

\CommentTok{# Let's take a look}
\NormalTok{word_df[index_genes, }\KeywordTok{c}\NormalTok{(}\StringTok{"PID"}\NormalTok{,}\StringTok{"Word"}\NormalTok{)]}
\end{Highlighting}
\end{Shaded}

\begin{verbatim}
##           PID         Word
## 171  32236213       carbon
## 198  32236213    pyrolysis
## 674  31763856 conductivity
## 5530 24727667 conductivity
## 6617 22362025       carbon
## 6765 21446719       carbon
\end{verbatim}

\end{document}
