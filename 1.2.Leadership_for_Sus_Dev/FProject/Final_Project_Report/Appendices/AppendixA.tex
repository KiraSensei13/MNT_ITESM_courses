% Appendix Template
\begin{landscape}

\chapter{The Green City Index} % Main appendix title

\label{AppendixA} % Change X to a consecutive letter; for referencing this appendix elsewhere, use \ref{AppendixX}

\begin{table}[th]
\caption{Supplementary Table: The Green City Index. Source: (Economist Intelligence Unit 2009; Venkatesh 2014)}
\begin{center}
\begin{tabular}{ >{\raggedright\arraybackslash}p{0.30\textwidth} >{\raggedright\arraybackslash}p{0.30\textwidth} >{\raggedright\arraybackslash}p{0.55\textwidth} >{\raggedright\arraybackslash}p{0.45 \textwidth} } 
\hline
Categories & Indicators & Description & Normalisation technique \\
\hline
1. Environmental governance &
1.1 Green action plan \linebreak
1.2. Green management \linebreak
1.3. Public participation in green policy &
1.1.1 An assessment of the ambitiousness and comprehensiveness of strategies to improve and monitor environmental performance. \linebreak
1.2.1 An assessment of the management of environmental issues and commitment to achieving international environmental standards. \linebreak
1.3.1 An assessment of the extent to which citizens may participate in environmental decision-making. &
1.1.2 Scored by Economist Intelligence Unit analysts on a scale of 0 to 10. \linebreak
1.2.2 Scored by Economist Intelligence Unit analysts on a scale of 0 to 10. \linebreak
1.3.2 Scored by Economist Intelligence Unit analysts on a scale of 0 to 10. \\
\hline
\label{tbl:theGreenCityIndex0}
\end{tabular}
\end{center}
\end{table}

\begin{table}[th]
\begin{center}
\begin{tabular}{ >{\raggedright\arraybackslash}p{0.30\textwidth} >{\raggedright\arraybackslash}p{0.30\textwidth} >{\raggedright\arraybackslash}p{0.55\textwidth} >{\raggedright\arraybackslash}p{0.45 \textwidth} }
\hline
Categories & Indicators & Description & Normalisation technique \\
\hline
2. Carbon dioxide &
2.1 Emissions \linebreak
2.2 Intensity \linebreak
2.3 Reduction strategy &
2.1.1 Total CO2 emissions, in tonnes per head. \linebreak
2.2.1 Total CO2 emissions, in grams per unit of real GDP (2000 base year). \linebreak
2.3.1 An assessment of the ambitiousness of CO2 emissions reduction strategy. &
2.1.1 Min-max. \linebreak
2.2.2 Min-max; lower benchmark of 1,000 grams inserted to prevent outliers. \linebreak
2.3.2 Scored by Economist Intelligence Unit analysts on a scale of 0 to 10. \\
\hline
3. Buildings &
3.1 Energy consumption of residential buildings \linebreak
3.2 Energy-efficient building standards \linebreak
3.3 Energy-efficient building initiatives &
3.1.1 Total final energy consumption in the residential sector, per square metre of residential floor space. \linebreak
3.2.1 An assessment the extensiveness of cities’ energy efficiency standards for buildings. \linebreak
3.3.1 An assessment of the extensiveness of efforts to promote energy efficiency of buildings. & 3.1.2 Min-max. \linebreak
3.2.2 Scored by Economist Intelligence Unit analysts on a scale of 0 to 10. \linebreak
3.3.2 Scored by Economist Intelligence Unit analysts on a scale of 0 to 10. \\
\hline
4. Transport &
4.1 Use of non-car transport \linebreak
4.2 Size of non-car transport network \linebreak
4.3 Green transport promotion \linebreak
4.4 Congestion reduction policies &
4.1.1 The total percentage of the working population travelling to work on public transport, by bicycle and by foot. \linebreak
4.2.1 Length of cycling lanes and the public transport network, in km per square metre of city area. \linebreak
4.3.1 An assessment of the extensiveness of efforts to increase the use of cleaner transport. \linebreak
4.4.1 An assessment of efforts to reduce vehicle traffic within the city. &
4.1.2 Converted to a scale of 0 to 10. \linebreak
4.2.2 Min-max. Upper benchmarks of 4 km/km2 and5 km/km2 inserted to prevent outliers. \linebreak
4.3.2 Scored by Economist Intelligence Unit analysts on a scale of 0 to 10. \linebreak
4.4.2 Scored by Economist Intelligence Unit analysts on a scale of 0 to 10. \\
\hline
\label{tbl:theGreenCityIndex1}
\end{tabular}
\end{center}
\end{table}

\begin{table}[th]
\begin{center}
\begin{tabular}{ >{\raggedright\arraybackslash}p{0.30\textwidth} >{\raggedright\arraybackslash}p{0.30\textwidth} >{\raggedright\arraybackslash}p{0.55\textwidth} >{\raggedright\arraybackslash}p{0.45 \textwidth} }
\hline
Categories & Indicators & Description & Normalisation technique \\
\hline
5. Water &
5.1 Water consumption \linebreak
5.2 Water system leakage \linebreak
5.3 Wastewater treatment \linebreak
5.4 Water efficiency and treatment policies &
5.1.1 Total annual water consumption, in cubic metres per head. \linebreak
5.2.1 Percentage of water lost in the water distribution system. \linebreak
5.3.1 Percentage of dwellings connected to the sewage system. \linebreak
5.4.1 An assessment of the comprehensiveness of measures \linebreak
to improve the efficiency of water usage and the treatment of wastewater. &
5.1.2 Min-max. \linebreak
5.2.2 Scored against an upper target of 5\%. \linebreak
5.3.2 Scored against an upper benchmark of 100\% and a lower benchmark of 80\%. \linebreak
5.4.2 Scored by Economist Intelligence Unit analysts on a scale of 0 to 10. \\
\hline
6. Waste and land use &
6.1 Municipal waste production \linebreak
6.2 Waste recycling \linebreak
6.3 Waste reduction policies \linebreak
6.4 Green land use policies &
6.1.1 Total annual municipal waste collected, in kg per head. \linebreak
6.2.1 Percentage of municipal waste recycled. \linebreak
6.3.1 An assessment of the extensiveness of measures to reduce the overall production of waste, and to recycle and reuse waste. \linebreak
6.4.1 An assessment of the comprehensiveness of policies to contain \linebreak
the urban sprawl and promote the availability of green spaces. &
6.1.2 Scored against an upper benchmark of 300 kg (EU target). \linebreak
A lower benchmark of 1,000 kg inserted to prevent outliers. \linebreak
6.2.2 Scored against an upper benchmark of 50\% (EU target). \linebreak
6.3.2 Scored by Economist Intelligence Unit analysts on a scale of 0 to 10. \linebreak
6.4.2 Scored by Economist Intelligence Unit analysts on a scale of 0 to 10. \\
\hline
\label{tbl:theGreenCityIndex2}
\end{tabular}
\end{center}
\end{table}

\begin{table}[th]
\begin{center}
\begin{tabular}{ >{\raggedright\arraybackslash}p{0.30\textwidth} >{\raggedright\arraybackslash}p{0.30\textwidth} >{\raggedright\arraybackslash}p{0.55\textwidth} >{\raggedright\arraybackslash}p{0.45 \textwidth} }
\hline
Categories & Indicators & Description & Normalisation technique \\
\hline
7. Energy &
7.1 Consumption \linebreak
7.2 Intensity \linebreak
7.3 Renewable energy consumption \linebreak
7.4 Clean and efficient energy policies &
7.1.1 Total final energy consumption, in gigajoules per head. \linebreak
7.2.1Total final energy consumption, in megajoules per unit \linebreak
of real GDP (in euros, base year 2000). \linebreak
7.3.1 The percentage of total energy derived from renewable sources, as a share of the city's total energy consumption, in terajoules. \linebreak
7.4.1 An assessment of the extensiveness of policies promoting the use of clean and efficient energy. &
7.1.2 Min-max. \linebreak
7.2.2 Min-max; lower benchmark of 8MJ/€GDP \linebreak
inserted to prevent outliers. \linebreak
7.3.2 Scored against an upper benchmark of 20\% (EU target). \linebreak
7.4.2 Scored by Economist Intelligence Unit analysts \linebreak
on a scale of 0 to 10.\\
\hline
8. Air quality &
8.1 Nitrogen dioxide \linebreak
8.2 Ozone \linebreak
8.3 Particulate Matter (PM) \linebreak
8.4 Sulphur dioxide \linebreak
8.5 Clean air policies &
8.1.1 Annual daily mean of NO2 emissions. \linebreak
8.2.1 Annual daily mean of O3 emissions. \linebreak
8.3.1 Annual daily mean of PM10 emissions. \linebreak
8.4.1 Annual daily mean of SO2 emissions. \linebreak
8.5.1 An assessment of the extensiveness of policies to improve air quality. &
8.1.2 Scored against a lower benchmark of 40 ug/m3 (EU target). \linebreak
8.2.2 Scored against a lower benchmark of 120 ug/m3 (EU target). \linebreak
8.3.2 Scored against a lower benchmark of 50 ug/m3 (EU target). \linebreak
8.4.2 Scored against a lower benchmark of 40 ug/m3 (EU target). \linebreak
8.5.2 Scored by Economist Intelligence Unit analysts on a scale of 0 to 10. \\
\hline
\label{tbl:theGreenCityIndex3}
\end{tabular}
\end{center}
\end{table}

\end{landscape}