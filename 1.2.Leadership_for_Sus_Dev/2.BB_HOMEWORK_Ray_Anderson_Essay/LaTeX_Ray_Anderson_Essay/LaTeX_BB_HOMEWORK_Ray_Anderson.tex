
%----------------------------------------------------------------------------------------%
% START LaTeX preamble

% define document type, font and paper size
\documentclass[11pt,a4paper]{article}

%----------------------------------------------------------------------------------------%
% IMPORT LaTeX packages

\usepackage{inputenc}
\usepackage[ngerman, english]{babel}
\usepackage{csquotes}
\usepackage{amsmath}
\usepackage{amssymb}
\usepackage{amsfonts}
\usepackage{graphicx}
\usepackage{wrapfig}
\usepackage[margin=1.25in]{geometry}

%----------------------------------------------------------------------------------------%
% IMPORT LaTeX packages to manange bibliography

% MLA, APA, or IEEE? - https://www.overleaf.com/learn/latex/Biblatex_citation_styles
\usepackage[style=apa, backend=biber]{biblatex}
\addbibresource{bibliography.bib}

%----------------------------------------------------------------------------------------%
% DEFINE header values

% define the cover page values
\title
{
    BB HOMEWORK: Ray Anderson Essay \\
    Video Analysis
}
\author
{
    Antonio Osamu Katagiri Tanaka \\
    A01212611
}
\date{\today}

%----------------------------------------------------------------------------------------%
% USER-DEFINED commands

% Keywords command
\providecommand{\keywords}[1]
{
    \\
    \\
    \small
    \textbf{\textit{Keywords:}} #1
}

%----------------------------------------------------------------------------------------%

\begin{document}

%----------------------------------------------------------------------------------------%
% CREATE the 1st page (cover page)

\maketitle

%----------------------------------------------------------------------------------------%
% DEFINE the abstract text & keywords

%\begin{abstract}
%    \emph
%    {
%        Lorem ipsum dolor sit amet, consectetur adipiscing elit, sed do eiusmod tempor incididunt ut labore et dolore magna aliqua. Ut enim ad minim veniam, quis nostrud exercitation ullamco laboris nisi ut aliquip ex ea commodo consequat. Duis aute irure dolor in reprehenderit in voluptate velit esse cillum dolore eu fugiat nulla pariatur. Excepteur sint occaecat cupidatat non proident, sunt in culpa qui officia deserunt mollit anim id est laborum.
%    }
%    \keywords{Lorem, ipsum, dolor, sit, amet}
%\end{abstract}
\clearpage

%----------------------------------------------------------------------------------------%
% CREATE a table of contents in a new page

%\tableofcontents
%\clearpage

%----------------------------------------------------------------------------------------%
% CREATE a list of figures and a list of tables in a new page

%\listoffigures
%\listoftables
%\clearpage

%----------------------------------------------------------------------------------------%
% DOCUMENT body starts here

\section{Video Analysis}\label{sec:intro}
Last year a paper was posted to Joule, addressing a new cheaper process of removing carbon from the Earth's atmosphere. \parencite{Keith2018} Now climate change can be solved, just give it some time and the problem will go away. However, it is known that for the past years that human-changes to the Earth's climate are already irreversible. \parencite{Solomon2009} The Earth is warming because of the persistent emissions of carbon dioxide (among other substances) into the atmosphere, preventing the excess energy from leaving the planet and therefore, causing an energy-imbalance. To prevent such warming, two actions can be taken: either the reduction the amount of energy coming into the Earth from the Sun or the removal of carbon dioxide from the atmosphere along with the increase of the amount of energy that the Earth radiates into space.

\cite{Keith2018} pictures a new technique of extracting carbon dioxide from the atmosphere and converting it into an energy-dense fuel. The new technique might work against climate change. But before the implementation of the new technique, let us understand the issues we are against. Climate scientist Susan Solomon issued a paper in \citefield{Solomon2009}{year} titled ``\citefield{Solomon2009}{title}". In summary, the conclusions of that paper states that the emissions of carbon due to human activities will alter the climate for over a thousand years in the future. To come to those conclusions, Solomon and her team examined how the climate is going to change, based on a collection of computer models known as Atmosphere–Ocean General Circulation Models (AOGCMs) and Earth System Models of Intermediate Complexity (EMICs). AOGCMs and EMICs describe the physics of how different parts of the Earth’s atmosphere talk to each other, and how air heats up.

Solomon's paper used a particular illustrative model called the Bern2.5CC EMIC to see how the atmospheric concentrations of $CO_2$ can change over the next thousand years. The authors found that about 40\% of the current peak $CO_2$ concentration will still be around in the 31st century. That is to say that if carbon missions are completely ceased today in the year 3000, the planet will still be nearly as warm as it was when $CO_2$ concentrations were at their peak in the 2000s. In other words, even if we stop emitting carbon today, the climate in a thousand years has already been altered.

The paper issued by Keith says that with their new technique, a tone of $CO_2$ can be extracted from the atmosphere for between \$94 USD and \$232 USD. However, according to the REN21's Global Status Report, 4.7 billion tons of $CO_2$ were emitted in 2017 within China and the European Union. Assuming that the process runs with the cost of \$94 USD per tone, the process would cost around 441.8 billion dollars per year (only to offset the emissions in the China and the European Union areas). In short, using Keith's technique and the economics presented in the paper, it is not possible to offset the world's emissions as they are. So we shall take action in a different way to address the impacts of climate change.

A more realistic way reduce the emissions that we need to offset is by changing our economy; and the recent \citefield{REN21}{title} \parencite{REN21} indicates that we are making progress in converting to a renewable-based energy economy and reducing future carbon emissions, but we still have a long way to go; particularly when it comes to transportation. As Anderson says in his TED talk ``\citefield{Anderson2009}{title}" \parencite{Anderson2009}, we shall ``[...] take from the earth only what can be renewed by the earth, naturally and rapidly [...]". And to do so Anderson proposes two actions: the transformation of our technologies and the halt of our growing affluence.

Anderson mentions Paul's well-known environmental impact equation (\ref{eq:1}) and amends it to explain his proposal.
\begin{equation} \label{eq:1}
EnvironmentalImpact = P A T_1
\end{equation}
\begin{flushright} \parencite{Hawken1993} \end{flushright}
where: P=population, A=affluence, T=technology

The first step is to replace $T_1$ in (\ref{eq:1}) with $T_2$. This implies the transformation of our current technologies in order to reduce (or completely halt) future carbon emissions. As Anderson states: ``[...] extractive must be replaced by renewable; linear by cyclical; fossilfuel energy by renewable energy, sunlight; wasteful by waste-free; and abusive by benign; and labor productivity by resource productivity." We need to revolutionize how we use energy in a general sense: transitioning away from fossil-fuels like coal and gas to renewable sources such as wind solar geothermal and hydro-power.
\begin{equation} \label{eq:2}
Environmental Impact = \dfrac{P A}{T_2}
\end{equation}
\begin{flushright} \parencite{Anderson2009} \end{flushright}

The second action is the reduction of our growing affluence by making the `` `A' a lowercase `a', suggesting that it is a means to an end, and that end is happiness". \parencite{Anderson2009} He states that we can reside in balance with the existing natural resources by being ``happy with less stuff".
\begin{equation} \label{eq:3}
Environmental Impact = \dfrac{P a}{T_2}
\end{equation}
\begin{flushright} \parencite{Anderson2009} \end{flushright}

%I agree that living with less stuff will contribute to the reduction of carbon emissions. However a better approach would be to transition from a ``greedy-growth" to a ``smart-growth", instead of a transition from ``greedy-growth" to ``no-growth". But that is a topic for a future essay.

Since climate change is a complex problem, where everyone contributes to the green-house emissions, solutions are poorly discussed. Our contributions to the carbon emissions happen on an individual level during our daily activities through a widespread network of interactions between the governments that we elect, the companies we choose, the banks we trust, etc. .

When articles are written, in regards on what we can do to mitigate climate change, they are focused very much on the individual actions we take, for instance: not eating meat \parencite{Milman2018} (or using public transport) and these are things we should be doing. However, considering that we are end-point users in a complex network, a more effective way of reducing our green-house gas emissions is to tackle the network interactions from the top-down, as Anderson implies in his TED talk.

While the industry make most green-house contributions, the majority carbon emissions come from energy production and energy consumption which includes electricity generation, heating, cooling, transportation, manufacturing, and other uses. A complete energy transformation can be a intimidating task but the REN21, Based at the UN Environment Programme, has produced a document every year since 2005 called the Global Gtatus Report (GSR). The GSR is written with contributions from energy experts from around the globe, assessing the renewable energy transition status and the current actions to bring green-house gas emissions down. 146 countries have goals for the implementation of renewables for power generation and those actions are working. In 2017, around a quarter of the energy used as electricity came from renewable sources. However, a lot is to be done regarding transportation and heating as the renewable implementation in those sectors is around 10\% for heating and 0.3\% for transportation. Only 42 countries have renewable energy goals for transportation and only 48 for heating. \parencite{REN21}

Let us support parties whose actions are prone to eradicate fossil-fuel subsidies. The common perception of renewable energy is that it is good for the environment, but too expensive. However, the cost per kilowatt hour of fossil-fuel power sources is similar to that of various renewable power sources. Most of the renewable technologies are either cost competitive or cheaper per kilowatt hour than fossil-fuel's. Fossil-fuels are subsidized significantly more than renewables. In 2016, 370 billion dollars were given in subsidies for fossil-fuel use and only 140 billion given to renewables. It is necessary to remove fossil-fuel subsidies and to increase renewable subsidies, to allow the energy transformation to take place. \parencite{REN21}

There are specific and impactful actions beyond the individual level to defy climate change. We've already damaged the planet in ways that will take several of decades to repair but, through collective action, we can reduce our green-house gas emissions and limit the damage we do. %The best time for action was 20 years ago, the second best time is now.

%----------------------------------------------------------------------------------------%
% PRINT bibliography/references in a new page

\clearpage
\printbibliography

%----------------------------------------------------------------------------------------%

\end{document}

%----------------------------------------------------------------------------------------%
