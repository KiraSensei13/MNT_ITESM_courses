%%%%%%%%%%%%%%%%%%%%%%%%%%%%%%%%%%%%%%%%%%%%%%%%%%%%%%%%%%%%%%%%%%%%%%
% writeLaTeX Example: A quick guide to LaTeX
%
% Source: Dave Richeson (divisbyzero.com), Dickinson College
% 
% A one-size-fits-all LaTeX cheat sheet. Kept to two pages, so it 
% can be printed (double-sided) on one piece of paper
% 
% Feel free to distribute this example, but please keep the referral
% to divisbyzero.com
% 
%%%%%%%%%%%%%%%%%%%%%%%%%%%%%%%%%%%%%%%%%%%%%%%%%%%%%%%%%%%%%%%%%%%%%%
% How to use writeLaTeX: 
%
% You edit the source code here on the left, and the preview on the
% right shows you the result within a few seconds.
%
% Bookmark this page and share the URL with your co-authors. They can
% edit at the same time!
%
% You can upload figures, bibliographies, custom classes and
% styles using the files menu.
%
% If you're new to LaTeX, the wikibook is a great place to start:
% http://en.wikibooks.org/wiki/LaTeX
%
%%%%%%%%%%%%%%%%%%%%%%%%%%%%%%%%%%%%%%%%%%%%%%%%%%%%%%%%%%%%%%%%%%%%%%

\documentclass[10pt,landscape]{article}
\usepackage{amssymb,amsmath,amsthm,amsfonts}
\usepackage{multicol,multirow}
\usepackage{calc}
\usepackage{ifthen}
\usepackage[landscape]{geometry}
\usepackage[colorlinks=true,citecolor=blue,linkcolor=blue]{hyperref}


\ifthenelse{\lengthtest { \paperwidth = 11in}}
    { \geometry{top=.5in,left=.5in,right=.5in,bottom=.5in} }
	{\ifthenelse{ \lengthtest{ \paperwidth = 297mm}}
		{\geometry{top=1cm,left=1cm,right=1cm,bottom=1cm} }
		{\geometry{top=1cm,left=1cm,right=1cm,bottom=1cm} }
	}
\pagestyle{empty}
\makeatletter
\renewcommand{\section}{\@startsection{section}{1}{0mm}%
                                {-1ex plus -.5ex minus -.2ex}%
                                {0.5ex plus .2ex}%x
                                {\normalfont\large\bfseries}}
\renewcommand{\subsection}{\@startsection{subsection}{2}{0mm}%
                                {-1explus -.5ex minus -.2ex}%
                                {0.5ex plus .2ex}%
                                {\normalfont\normalsize\bfseries}}
\renewcommand{\subsubsection}{\@startsection{subsubsection}{3}{0mm}%
                                {-1ex plus -.5ex minus -.2ex}%
                                {1ex plus .2ex}%
                                {\normalfont\small\bfseries}}
\makeatother
\setcounter{secnumdepth}{0}
\setlength{\parindent}{0pt}
\setlength{\parskip}{0pt plus 0.5ex}

% -----------------------------------------------------------------------

\title{Quick Guide to LaTeX}

\begin{document}

\raggedright
\footnotesize

\begin{center}
     \Large{\textbf{Algorithms, Evidence, and Data Science Cookbook}} \\
\end{center}
\begin{multicols}{3}
\setlength{\premulticols}{1pt}
\setlength{\postmulticols}{1pt}
\setlength{\multicolsep}{1pt}
\setlength{\columnsep}{2pt}

% -----------------------------------------------------------------------
\section{Part I: Classic Statistical Inference} % -----------------------
% -----------------------------------------------------------------------

* \textbf{Population:} the entire group

* \textbf{Sample:} a subset of the population

* \textbf{Mean:} 
$\mu$ is the mean of the population; $\bar{x}$ is the mean of the sample
$$ \underset{i=1}{\overset{n}{\frac{1}{n}\sum }}x_i $$

* \textbf{Variance:} the dispersion around the mean
\begin{multicols}{2}
Variance of a population:
$$ \sigma^2 = \underset{i=1}{\overset{n}{\frac{1}{n}\sum }} {\left(x_i - \mu \right)}^2 $$

\pagebreak 
Variance of a sample:
$$ s^2 = \underset{i=1}{\overset{n}{\frac{1}{n}\sum }} {\left(x_i - \bar{x} \right)}^2 $$
\end{multicols}

* \textbf{Standard Deviation:} square root of the variance

* \textbf{Standard Error:} an estimate of the standard deviation of the sampling distribution
\begin{multicols}{2}
For a mean:
$$ se(\bar{x}) = \frac{s}{\sqrt{n}} $$

\pagebreak 
For the difference between two means:
$$ se(\bar{x_1}, \bar{x_2}) = \sqrt{\frac{{s_1}^2}{n_1} + \frac{{s_2}^2}{n_2}} $$
\end{multicols}

\subsection{T-test, one-sample}
* null hypothesis $H_o : \mu = \mu_0$ \\
* alternative hypothesis $H_a : \mu \{=, > or <\} \mu_0$ \\
* $t-statistic t$ standarices the difference between $\bar{x}$ and $\mu_0$
$$ t = \frac{\bar{x} - \mu_0}{se(\bar{x})} $$
degrees of freedom $df = n - 1$ \\
* $p-value$: probability that $\bar{x}$ was obtained by chance given $\mu_0 = \mu$. \\
* \textbf{algorithm:} read the t-distribution critical values (chart) for the $p-value$ using $t$ and $df$ \\
if($p-value < \alpha$)\{ reject $H_o$ and accept $H_a$ \} \\
else \{ cant reject $H_o$ \} \\
* $\alpha$ is the predetermined value of significance (usually 0.05) \\
* if($t$ is of the 'wrong' sign){$p-value = 1 - {p-value}_{chart}$}

\subsection{paired two-sample t-test}
each value of one group corresponds to a value in the other group\\
* \textbf{algorithm:} subtract the values for each sample to get one set of values and use $\mu_0$ to perform a one-sample t-test\\

\subsection{unpaired two-sample t-test}
the two populations are independent \\
* $H_o : \mu_1 = \mu_2$ \\
* $H_a : \mu_1 \{=, > or <\} \mu_2$ \\
* $t-statistic t$
$$ t = \frac{\bar{x_1} - \bar{x_2}}{se(\bar{x_1}, \bar{x_2})} $$
degrees of freedom $df = (n_1 - 1) + (n_2 - 1)$ \\
* \textbf{algorithm:} same as in one-sample t-test \\
* double the $p-value$ for $H_a : \mu_1 \neq \mu_2$ \\

\medskip 
* \textbf{Type I error $\alpha$:} probability of rejecting a true $H_o$ \\
* \textbf{Type II error $\beta$:} probability of failing to reject a false $H_o$ \\

% -----------------------------------------------------------------------
\subsection{Algorithms and Inference} % -------------------------------

* \textbf{Algorithm:} set of data probability-steps to produce an estimator \\
* \textbf{Inference:} measuring the uncertainty around the estimator \\

\emph{e.g.:} $\bar{x}$ the algorithm, while $se(\bar{x})$ is the inference 

% -----------------------------------------------------------------------
\subsubsection{A Regression Example}

any regression is a conditional mean $\hat{Y_i} = E(Y_i|X_i)$ \\
* $Y:$ response variable \\
* $X:$ covariate/predictor/feature \\
* predicted values = fitted curve given $x$:
$$\hat{Y}(x) = \hat{\beta_0} + \hat{\beta_1} x$$

% -----------------------------------------------------------------------
\subsubsection{Hypothesis Testing}

% -----------------------------------------------------------------------
\subsubsection{Notes}

% -----------------------------------------------------------------------
\subsection{Frequentist Inference} % ------------------------------------

% -----------------------------------------------------------------------
\subsubsection{Frequentism in Practice}

% -----------------------------------------------------------------------
\subsubsection{Frequentist Optimality}

% -----------------------------------------------------------------------
\subsubsection{Notes and Details}

% -----------------------------------------------------------------------
\subsection{Bayesian Inference} % ---------------------------------------

% -----------------------------------------------------------------------
\subsubsection{Two Examples}

% -----------------------------------------------------------------------
\subsubsection{Uninformative Prior Distributions}

% -----------------------------------------------------------------------
\subsubsection{Flaws in Frequentist Inference}

% -----------------------------------------------------------------------
\subsubsection{A Bayesian/Frequentist Comparison List}

% -----------------------------------------------------------------------
\subsubsection{Notes and Details}

% -----------------------------------------------------------------------
\subsection{Fisherian Inference and Maximum Likelihood Estimation} % ----

% -----------------------------------------------------------------------
\subsubsection{Likelihood and Maximum Likelihood}

% -----------------------------------------------------------------------
\subsubsection{Fisher Information and the MLE}

% -----------------------------------------------------------------------
\subsubsection{Conditional Inference}

% -----------------------------------------------------------------------
\subsubsection{Permutation and Randomization}

% -----------------------------------------------------------------------
\subsubsection{Notes and Details}

% -----------------------------------------------------------------------
\subsection{Parametric Models and Exponential Families} % ---------------

% -----------------------------------------------------------------------
\subsubsection{Univariate Families}

% -----------------------------------------------------------------------
\section{} % ------------------------------------------------------------
\hrule
Osamu Katagiri - A01212611, \href{https://www.linkedin.com/in/osamu-katagiri-84b2b940/}{linkedin.com/osamu-katagiri/}
\end{multicols}

\end{document}
