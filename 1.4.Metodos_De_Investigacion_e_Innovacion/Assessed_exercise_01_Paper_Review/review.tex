\documentclass{article}

\usepackage[utf8]{inputenc}
\usepackage{textcomp}
\usepackage{amsmath}
\usepackage{amssymb}
\usepackage{pdfpages}

% MLA, APA, or IEEE? - https://www.overleaf.com/learn/latex/Biblatex_citation_styles
\usepackage[style=apa, backend=biber]{biblatex}
\addbibresource{bibliography.bib}

\newcommand{\dbx}[2]{\rule[-#2]{0cm}{#2}{\hspace{#1}}}
\newcommand{\bx}[1]{\fbox{\rule{0cm}{0.08in}{\hspace{#1}}}}
\newcommand{\option}[1]{#1\bx{.2in}}
\newcommand{\five}[0]{\bx{.15in}\bx{.15in}\bx{.15in}\bx{.15in}\bx{.15in}}
\newcommand{\YESNO}[0]{YES \five NO}
\newcommand{\degree}[2]{#1\five #2}

\title{Tecnol\'ogico de Monterrey\\
Paper Review Form\thanks{
Format shamelessly stolen from Alan Bundy.}}
\author{}
\date{}

\begin{document}
\maketitle
\vspace{-2cm}

\begin{description}
\item Nombre: \boxed{Antonio\ Osamu\ Katagiri\ Tanaka} \hspace{0.30in}
Matr\'{\i}cula: \boxed{A01212611}
\item Materia: \boxed{M\acute{e}todos\ de\ investigaci\acute{o}n\ e\ innovaci\acute{o}n} \hspace{0.30in}
Clave: \boxed{GI5000.1}
\item Art\'{\i}culo: \boxed{
  \!\begin{aligned}
  Nano-electric\ field\ sensor\ based\ on\ Two\ Dimensional \\ Photonic\ Crystal\ resonator
  \end{aligned}
} \\ \parencite{Rajasekar2018}
\end{description}

\section*{Instructions} 

{\small Read the paper, complete a copy of this form and return it,
as already indicated, to the module lecturer. You ought to complete each 
section, otherwise you will not be given full credit.}

\section{Content of Research}
{\small What kind of contribution did the paper attempt to make?
Various possible kinds of contribution are listed below.  Tick
each of those that applies and justify your assessment. Use the
comments section for extended discussion.}
\begin{verse}

%CHECKED
%\makebox[0pt][l]{$\square$}\raisebox{.15ex}{\hspace{0.1em}$\checkmark$}

%UNCHECKED
%$\square$

\makebox[0pt][l]{$\square$}\raisebox{.15ex}{\hspace{0.1em}$\checkmark$}
1. Describes a new technique.  \\
$\square$
2. Extends or improves an existing technique.  \\
$\square$
3. Establishes properties of a technique or relations between two
or more techniques either empirically or theoretically.  \\
$\square$
4. Describes a new application of a technique.  \\
$\square$
5. Tests the psychological validity of a technique.  \\
$\square$
6. Combines several techniques into a system.  \\
$\square$
7. Identifies and motivates a new problem.  \\
$\square$
8. Serves a tutorial role, {\em e.g.}~a survey.  \\
$\square$
9. Other (please specify).  \\
\end{verse}
Comments: \\
\boxed{
  \!\begin{aligned}
  Rajasekar's paper 
  \end{aligned}
}

\section{Correctness and Completeness of Paper}
{\small Assess the correctness and completeness of the paper by
ticking which of the following apply. Justify your answers in
the comments section.}
\begin{verse}

%CHECKED
%\makebox[0pt][l]{$\square$}\raisebox{.15ex}{\hspace{0.1em}$\checkmark$}

%UNCHECKED
%$\square$

$\square$
1. The methodology employed by the authors is not sensible.  \\
$\square$
2. There are major technical errors in the paper.  \\
$\square$
3. There are major omissions in the paper.  \\
$\square$
4. The reported research is not in a completed state.  \\
$\square$
5. Related research is not adequately compared.  \\
$\square$
6. The paper is deficient in some other way (please specify).  \\
\end{verse}
Comments: \\
\boxed{}

\section{Significance of Research}
Assess the significance of the reported research by answering the
following questions. Justify your answers.
\begin{enumerate}
% \item Has this or very similar work been published before either by
% the author or by someone else? \\
% \framebox[4.4in][l]{\raisebox{0.15in}[0in][0.3in]{}}

\item   What hypotheses are tested by the reported research? \\
\boxed{}

\item   What are the outcomes of this hypothesis testing? \\
\boxed{}

\item   Assess the importance of the reported research on a scale of 1
(trivial) to 5 (major breakthrough).  \\
\boxed{}
\end{enumerate}

% \section{Presentation of Paper}
% Assess how well the paper is presented by ticking the statements
% that you agree with. Justify your answers.
% \begin{verse}
% $\Box$ 1. The paper is easy for the intended audience to read  \\
% $\Box$ 2. The paper is unambiguous and precise \\
% $\Box$ 3. The paper is well organised \\
% $\Box$ 4. It is clear what points are being made by the paper \\
% $\Box$ 5. The paper is self-contained \\
% \end{verse}
% Comments: \\
% \framebox[5.5in][l]{\raisebox{0.15in}[0in][1.5in]{}}

\section{Precis}
Describe the message of the paper in your own words. Use 100-200
words. \\
\boxed{}

\clearpage
\printbibliography

\includepdf[pages=-]{Nanoelectric_field_sensor}

\end{document}

