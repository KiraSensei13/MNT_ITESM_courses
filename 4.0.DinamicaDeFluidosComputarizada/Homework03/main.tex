\documentclass{article}
\usepackage[utf8]{inputenc}

\title{Homework No.3}
\author{Osamu Katagiri-Tanaka : A01212611}
\date{\today}

% import math symbols
\usepackage{amsmath, esint}
\usepackage{cancel}

% import code snippets
\usepackage{listings}
\usepackage{xcolor}
\definecolor{codegreen}{rgb}{0,0.6,0}
\definecolor{codegray}{rgb}{0.5,0.5,0.5}
\definecolor{codepurple}{rgb}{0.58,0,0.82}
\definecolor{backcolour}{rgb}{0.99,0.99,0.96}
\lstdefinestyle{mystyle}{
    backgroundcolor=\color{backcolour},   
    commentstyle=\color{codegreen},
    keywordstyle=\color{magenta},
    numberstyle=\tiny\color{codegray},
    stringstyle=\color{codepurple},
    basicstyle=\ttfamily\small,
    breakatwhitespace=false,         
    breaklines=true,                 
    captionpos=b,                    
    keepspaces=true,                 
    numbers=left,                    
    numbersep=5pt,                  
    showspaces=false,                
    showstringspaces=false,
    showtabs=false,                  
    tabsize=2
}
\lstset{style=mystyle}

% import continuous lists
\usepackage{enumitem}

% format margins and paper size
\usepackage{geometry}
\geometry{
	paper         = a4paper, % Change to letterpaper for US letter
	inner         = 2.5cm,   % Inner margin
	outer         = 2.5cm,   % Outer margin
	bindingoffset = 0.5cm,   % Binding offset
	top           = 1.5cm,   % Top margin
	bottom        = 1.5cm    % Bottom margin
}

% import figure handler
\usepackage{graphicx}

% import references handler
\usepackage[
    style     = ieee,         % references format style
    backend   = biber,        % choose the processing program
    natbib    = true,         % enable additional reference formats
    citestyle = numeric-comp, % enable multiple citations
    sortcites = true,         % sort references in multiple citations
    sorting   = nyt           % sort the reference table
]{biblatex}
\addbibresource{references.bib}

% Note that ‘d’ in the differential is conventionally set in roman.
\newcommand{\ud}{\,\mathrm{d}}

% Paragraph spacing
\setlength{\parskip}{0.2cm}           % spacing between paragraphs
\renewcommand{\baselinestretch}{1.25} % spacing between lines

\begin{document}

\maketitle

\section{Problem Statement}

The equation of conservation of chemical species under a chemical reaction of
decomposition can be represented with the PDE given below.

$$ \frac{\partial C}{\partial t} = \vec{\nabla} \cdot (D \vec{\nabla} C) - \vec{v} \cdot \vec{\nabla} C - k C^n $$

If a tubular catalytic chemical reactor initially filled with an inert solvent $(C = 0)$ is fed by a stream of component ``A" with a concentration of $1 kmol/m^3$ $(C = 1)$ and speed of $1 m/s$ $(v = 1)$, calculate the distribution of ``A" across the reactor and as a function of time $C(x, t)$. The dispersion coefficient of the component ``A" is $0.02 m^2/s$ $(D = 0.01)$, the kinetic decomposition coefficient $0.05 s^{-1}$ $(k = 0.05)$. The chemical decomposition kinetics is first order $(n = 1)$.

\section{Sketch}

\section{Assumptions and Approximations}

% You can assume 1-m long reactor

\section{Physical constants}

\section{Physical Transport or Thermodynamic Properties}

\section{Calculations}

The molar balance in axial direction for a 1D flow can be written as:

$$ \frac{\partial C}{\partial t} = D \frac{\partial^2 C}{\partial x^2} - v \frac{\partial C}{\partial x} - k C^n $$

The initial condition $IC$ is:
$$ \left. C \right|_{t = 0} = 0 \textrm{, } 0 \leq x \leq 1 $$

The boundary conditions $BCs$ are:
$$ \left. C \right|_{x = 0} = 1 \textrm{, } t > 0 $$
$$ \left. \frac{\partial C}{\partial t} \right|_{x = L} = 0 \textrm{, } t \geq 0 $$

\subsection{PDEPE solver}

The built in function PDEPE, solves a general problem of a 1-D (parabolic or elliptic) partial differential equation, for a Cartesian, cylindrical or spherical coordinates of the from:

$$ c \left( x, t, u, \frac{\partial u}{\partial x} \right) \frac{\partial u}{\partial t} = \frac{1}{x^m} \frac{\partial}{\partial x} \left( x^m f \left( x, t, u, \frac{\partial u}{\partial x} \right) \right) + s \left( x, t, u, \frac{\partial u}{\partial x} \right) $$

Where, \\
$\displaystyle m = 0$ represents the symmetry of the problem (0 for slab, 1 for cylindrical, or 2 for spherical) \\
$\displaystyle c = 1$ is a diagonal matrix \\
$\displaystyle f = D \frac{\partial u}{\partial x}$ is the flux term\\
$\displaystyle s = - v \frac{\partial u}{\partial x} - k u^n$ is the source term \\
$c$, $f$, and $s$ correspond to coefficients in the standard PDE equation form expected by pdepe. These coefficients are coded in terms of the input variables $x$, $t$, $u$, and $dudx$. Listing \ref{DiffusionPDEfun} implements a function that calculates the values of the coefficients $c$, $f$, and $s$.

\begin{lstlisting}[language=Matlab, caption=PDE function for equations, label=DiffusionPDEfun]
function [c, f, s] = DiffusionPDEfun(x, t, u, dudx, P)
    % Parameters
    D  = P(1);
    k  = P(3);
    vo = P(4);
    
    % PDE
    c = 1;
    f = D .* dudx;
    s = - k * u - vo * dudx;
end
\end{lstlisting}

PDEPE requires an `initial condition function', which is defined as a function that defines the initial condition. For $t = t_o = 0$ and all $x$, the solution satisfies the initial condition of the form:

$$ u(x, t_o) = u_o(x) $$

PDEPE calls the initial condition function with an argument $x$, which evaluates the initial values for the solution at $x$ in vector $u_o$. The number of elements in $u_o$ is equal to the number of equations. Listing \ref{DiffusionICfun} implements the constant initial condition.

\begin{lstlisting}[language=Matlab, caption=Initial condition function, label=DiffusionICfun]
function u0 = DiffusionICfun(x, P)
    % u0 = u0(x)
    u0 = 0;
end
\end{lstlisting}

The third function required by the PDEPE solver is the `boundary condition function'. The boundary condition function spefifies the boundary conditions for all $t$, the solution satisfy the boundary condition of the form:

$$ p(x, t, u) + q(x, t) f \left( x, t, u, \frac{\partial u}{\partial x} \right) = 0 $$

Listing \ref{DiffusionBCfun} implements a function that defines the terms $p$ and $q$ of the boundary conditions. $u1$ is the approximate solution of the left boundary, $ur$ is the approximate solution of the right boundary, $pl$ and $ql$ are vectors corresponding to $p$ and $q$ evaluated at $xl$, and $pr$ and $qr$ are vectors corresponding to $p$ and $q$ evaluated at $xr$.

\begin{lstlisting}[language=Matlab, caption=Boundary condition function, label=DiffusionBCfun]
function [pl, ql, pr, qr] = DiffusionBCfun(xl, u1, xr, ur, t, P)
    % BCs: No flux boundary at the right boundary and constant
    % concentration on the left boundary
    c0 = P(2);
    pl = u1 - c0;
    ql = 0;
    pr = 0;
    qr = 1;
end
\end{lstlisting}

\subsection{FEATool solver}

\section{Discussion}

\printbibliography[title={References}]
\end{document}