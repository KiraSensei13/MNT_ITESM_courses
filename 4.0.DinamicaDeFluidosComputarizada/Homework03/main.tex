\documentclass{article}
\usepackage[utf8]{inputenc}

\title{Homework No.3}
\author{Osamu Katagiri-Tanaka : A01212611}
\date{\today}

% import math symbols
\usepackage{amsmath, esint}
\usepackage{cancel}

% import continuous lists
\usepackage{enumitem}

% format margins and paper size
\usepackage{geometry}
\geometry{
	paper         = a4paper, % Change to letterpaper for US letter
	inner         = 2.5cm,   % Inner margin
	outer         = 2.5cm,   % Outer margin
	bindingoffset = 0.5cm,   % Binding offset
	top           = 1.5cm,   % Top margin
	bottom        = 1.5cm    % Bottom margin
}

% import figure handler
\usepackage{graphicx}

% import references handler
\usepackage[
    style     = ieee,         % references format style
    backend   = biber,        % choose the processing program
    natbib    = true,         % enable additional reference formats
    citestyle = numeric-comp, % enable multiple citations
    sortcites = true,         % sort references in multiple citations
    sorting   = nyt           % sort the reference table
]{biblatex}
\addbibresource{references.bib}

% Note that ‘d’ in the differential is conventionally set in roman.
\newcommand{\ud}{\,\mathrm{d}}

% Paragraph spacing
\setlength{\parskip}{0.2cm}           % spacing between paragraphs
\renewcommand{\baselinestretch}{1.25} % spacing between lines

\begin{document}

\maketitle

\section{Problem Statement}

The equation of conservation of chemical species under a chemical reaction of
decomposition can be represented with the PDE given below.

$$ \frac{\partial C}{\partial t} = \vec{\nabla} \cdot (D \vec{\nabla} C) - \vec{v} \cdot \vec{\nabla} C - k C^n $$

If a tubular catalytic chemical reactor initially filled with an inert solvent $(C = 0)$ is fed by a stream of component ``A" with a concentration of $1 kmol/m^3$ $(C = 1)$ and speed of $1 m/s$ $(v = 1)$, calculate the distribution of ``A" across the reactor and as a function of time $C(x, t)$. The dispersion coefficient of the component ``A" is $0.02 m^2/s$ $(D = 0.01)$, the kinetic decomposition coefficient $0.05 s^{-1}$ $(k = 0.05)$. The chemical decomposition kinetics is first order $(n = 1)$.

\section{Sketch}

\section{Assumptions and Approximations}

% You can assume 1-m long reactor

\section{Physical constants}

\section{Physical Transport or Thermodynamic Properties}

\section{Calculations}

\subsection{PDEPE solver}

The molar balance in axial direction for a 1D flow can be written as:

$$ \frac{\partial C}{\partial t} = D \frac{\partial^2 C}{\partial x^2} - v \frac{\partial C}{\partial x} - k C^n $$

The initial condition $IC$ is:

$$ \left. C \right|_{t = 0} = 0 \textrm{, } 0 \leq x \leq 1 $$

The boundary conditions $BCs$ are:

$$ \left. C \right|_{x = 0} = 1 \textrm{, } t > 0 $$
$$ \left. \frac{\partial C}{\partial t} \right|_{x = L} = 0 \textrm{, } t \geq 0 $$

\subsection{FEATool solver}

\section{Discussion}

\printbibliography[title={References}]
\end{document}