%% AMS-LaTeX Created with the Wolfram Language for Students - Personal Use Only : www.wolfram.com

\documentclass{article}
\usepackage{amsmath, amssymb, graphics, setspace}

\newcommand{\mathsym}[1]{{}}
\newcommand{\unicode}[1]{{}}

\newcounter{mathematicapage}
\begin{document}

I.a\\
A=\(\left(
\begin{array}{ccc}
 1 & 0 & 0 \\
 2 & 1 & 3 \\
 0 & 0 & 1 \\
\end{array}
|
\begin{array}{ccc}
 1 & 0 & 0 \\
 0 & 1 & 0 \\
 0 & 0 & 1 \\
\end{array}
\right)\Rightarrow\) \(R_2-2R_1\rightarrow R_2\Rightarrow\) \(\left(
\begin{array}{ccc}
 1 & 0 & 0 \\
 0 & 1 & 3 \\
 0 & 0 & 1 \\
\end{array}
|
\begin{array}{ccc}
 1 & 0 & 0 \\
 -2 & 1 & 0 \\
 0 & 0 & 1 \\
\end{array}
\right)\Rightarrow\) \(R_2-3R_3\rightarrow R_2\Rightarrow\) \(\left(
\begin{array}{ccc}
 1 & 0 & 0 \\
 0 & 1 & 0 \\
 0 & 0 & 1 \\
\end{array}
|
\begin{array}{ccc}
 1 & 0 & 0 \\
 -2 & 1 & -3 \\
 0 & 0 & 1 \\
\end{array}
\right)\)\\
\\
I.b\\
B=\(\left(
\begin{array}{ccc}
 1 & 1 & 1 \\
 1 & 2 & 2 \\
 1 & 2 & 3 \\
\end{array}
|
\begin{array}{ccc}
 1 & 0 & 0 \\
 0 & 1 & 0 \\
 0 & 0 & 1 \\
\end{array}
\right)\Rightarrow\) \(\begin{array}{c}
 R_2-R_1\rightarrow R_2 \\
 R_3-R_1\rightarrow R_3 \\
\end{array}
\Rightarrow\) \(\left(
\begin{array}{ccc}
 1 & 1 & 1 \\
 0 & 1 & 1 \\
 0 & 1 & 2 \\
\end{array}
|
\begin{array}{ccc}
 1 & 0 & 0 \\
 -1 & 1 & 0 \\
 -1 & 0 & 1 \\
\end{array}
\right)\Rightarrow\) \(\begin{array}{c}
 R_1-R_2\rightarrow R_1 \\
 R_3-R_2\rightarrow R_3 \\
\end{array}
\Rightarrow\) \(\left(
\begin{array}{ccc}
 1 & 0 & 0 \\
 0 & 1 & 1 \\
 0 & 0 & 1 \\
\end{array}
|
\begin{array}{ccc}
 2 & -1 & 0 \\
 -2 & 1 & 0 \\
 0 & -1 & 1 \\
\end{array}
\right)\Rightarrow\) \(R_2-R_3\rightarrow R_2\Rightarrow\) \(\left(
\begin{array}{ccc}
 1 & 0 & 0 \\
 0 & 1 & 0 \\
 0 & 0 & 1 \\
\end{array}
|
\begin{array}{ccc}
 2 & -1 & 0 \\
 -1 & 2 & -1 \\
 0 & -1 & 1 \\
\end{array}
\right)\)\\
\\
I.c\\
C=\(\left(
\begin{array}{ccc}
 -\frac{4}{\sqrt{3}} & 0 & 0 \\
 0 & -\frac{2}{3} & 0 \\
 0 & 0 & e^2 \\
\end{array}
\right)\Rightarrow\) \(C^{-1}=\text{diag}\left(\frac{1}{C_1},\frac{1}{C_2},\ldots ,\frac{1}{C_n}\right)\Rightarrow\) \(\left(
\begin{array}{ccc}
 -\frac{\sqrt{3}}{4} & 0 & 0 \\
 0 & -\frac{3}{2} & 0 \\
 0 & 0 & \frac{1}{e^2} \\
\end{array}
\right)\)\\
\\
I.d\\
D=\(\left(
\begin{array}{ccc}
 \pi  & \pi  & \pi  \\
 2 & 2 & 2 \\
 1 & 1 & 1 \\
\end{array}
|
\begin{array}{ccc}
 1 & 0 & 0 \\
 0 & 1 & 0 \\
 0 & 0 & 1 \\
\end{array}
\right)\Rightarrow\) \(\begin{array}{c}
 \frac{1}{\pi }R_1\rightarrow R_1 \\
 \frac{1}{2}R_2\rightarrow R_2 \\
\end{array}
\Rightarrow\) \(\left(
\begin{array}{ccc}
 1 & 1 & 1 \\
 1 & 1 & 1 \\
 1 & 1 & 1 \\
\end{array}
|
\begin{array}{ccc}
 \frac{1}{\pi } & 0 & 0 \\
 0 & \frac{1}{2} & 0 \\
 0 & 0 & 1 \\
\end{array}
\right)\Rightarrow\) \(\begin{array}{c}
 R_2-R_1\rightarrow R_2 \\
 R_3-R_1\rightarrow R_3 \\
\end{array}
\Rightarrow\) \(\left(
\begin{array}{ccc}
 1 & 1 & 1 \\
 0 & 0 & 0 \\
 0 & 0 & 0 \\
\end{array}
|
\begin{array}{ccc}
 \frac{1}{\pi } & 0 & 0 \\
 -\frac{1}{\pi } & \frac{1}{2} & 0 \\
 -\frac{1}{\pi } & 0 & 1 \\
\end{array}
\right)\Rightarrow\) D is not invertible since: a) the reduced row echelon of D contains at least one zero-row; and b) the matrix on the left-hand
side is not precisely the identity matrix.\\


II

\begin{doublespace}
\noindent\(\pmb{\text{varE}=\left(
\begin{array}{ccc}
 4 & 5 & 6 \\
 0 & 7 & 8 \\
 0 & 0 & 9 \\
\end{array}
\right);}\\
\pmb{\text{Inverse}[\text{varE}];}\\
\pmb{\text{MatrixForm}[\%]}\)
\end{doublespace}

\begin{doublespace}
\noindent\(\left(
\begin{array}{ccc}
 \frac{1}{4} & -\frac{5}{28} & -\frac{1}{126} \\
 0 & \frac{1}{7} & -\frac{8}{63} \\
 0 & 0 & \frac{1}{9} \\
\end{array}
\right)\)
\end{doublespace}

III.a

\begin{doublespace}
\noindent\(\pmb{\text{varA}=\left(
\begin{array}{ccc}
 1 & 0 & 0 \\
 2 & 1 & 3 \\
 0 & 0 & 1 \\
\end{array}
\right);}\\
\pmb{\text{Inverse}[\text{varA}];}\\
\pmb{\text{MatrixForm}[\%]}\)
\end{doublespace}

\begin{doublespace}
\noindent\(\left(
\begin{array}{ccc}
 1 & 0 & 0 \\
 -2 & 1 & -3 \\
 0 & 0 & 1 \\
\end{array}
\right)\)
\end{doublespace}

III.b

\begin{doublespace}
\noindent\(\pmb{\text{varB}=\left(
\begin{array}{ccc}
 1 & 1 & 1 \\
 1 & 2 & 2 \\
 1 & 2 & 3 \\
\end{array}
\right);}\\
\pmb{\text{Inverse}[\text{varB}];}\\
\pmb{\text{MatrixForm}[\%]}\)
\end{doublespace}

\begin{doublespace}
\noindent\(\left(
\begin{array}{ccc}
 2 & -1 & 0 \\
 -1 & 2 & -1 \\
 0 & -1 & 1 \\
\end{array}
\right)\)
\end{doublespace}

III.c

\begin{doublespace}
\noindent\(\pmb{\text{varC}=\left(
\begin{array}{ccc}
 -\frac{4}{\sqrt{3}} & 0 & 0 \\
 0 & -\frac{2}{3} & 0 \\
 0 & 0 & e^2 \\
\end{array}
\right);}\\
\pmb{\text{Inverse}[\text{varC}];}\\
\pmb{\text{MatrixForm}[\%]}\)
\end{doublespace}

\begin{doublespace}
\noindent\(\left(
\begin{array}{ccc}
 -\frac{\sqrt{3}}{4} & 0 & 0 \\
 0 & -\frac{3}{2} & 0 \\
 0 & 0 & \frac{1}{e^2} \\
\end{array}
\right)\)
\end{doublespace}

III.d\\
Matrix is singular.

\begin{doublespace}
\noindent\(\pmb{\text{varD}=\left(
\begin{array}{ccc}
 \pi  & \pi  & \pi  \\
 2 & 2 & 2 \\
 1 & 1 & 1 \\
\end{array}
\right);}\\
\pmb{\text{Inverse}[\text{varD}];}\)
\end{doublespace}

IV.a\\
\(\(A^{-1}=\)\) \(\(\frac{1}{1(8)-2(2)}\left(
\begin{array}{cc}
 8 & -2 \\
 -2 & 1 \\
\end{array}
\right)=\)\) \(\(\frac{1}{4}\left(
\begin{array}{cc}
 8 & -2 \\
 -2 & 1 \\
\end{array}
\right)=\)\) \(\(\left(
\begin{array}{cc}
 2 & -\frac{1}{2} \\
 -\frac{1}{2} & \frac{1}{4} \\
\end{array}
\right)\)\)

IV.b\\
\(\(B^{-1}=\)\) \(\(\frac{1}{0(0)-2(3)}\left(
\begin{array}{cc}
 0 & -2 \\
 -3 & 0 \\
\end{array}
\right)=\)\) \(\(-\frac{1}{6}\left(
\begin{array}{cc}
 0 & -2 \\
 -3 & 0 \\
\end{array}
\right)=\)\) \(\(\left(
\begin{array}{cc}
 0 & \frac{1}{3} \\
 \frac{1}{2} & 0 \\
\end{array}
\right)\)\)

IV.c\\
\(\(C^{-1}=\)\) \(\(\frac{1}{2(0)-0(4)}\left(
\begin{array}{cc}
 0 & 0 \\
 -4 & 2 \\
\end{array}
\right)\Rightarrow\)\) ad-bc$\neq $0, therefore \(\(C^{-1}\)\) does not exist.

IV.d\\
\(\(D^{-1}=\)\) \(\(\frac{1}{\text{Cos}[\theta ](\text{Cos}[\theta ])-(-\text{Sin}[\theta ])(\text{Sin}[\theta ])}\left(
\begin{array}{cc}
 \text{Cos}[\theta ] & \text{Sin}[\theta ] \\
 -\text{Sin}[\theta ] & \text{Cos}[\theta ] \\
\end{array}
\right)=\)\) \(\(\frac{1}{\text{Cos}[\theta ]^2+\text{Sin}[\theta ]^2}\left(
\begin{array}{cc}
 \text{Cos}[\theta ] & \text{Sin}[\theta ] \\
 -\text{Sin}[\theta ] & \text{Cos}[\theta ] \\
\end{array}
\right)=\)\) \(\(\left(
\begin{array}{cc}
 \frac{\text{Cos}[\theta ]}{\text{Cos}[\theta ]^2+\text{Sin}[\theta ]^2} & \frac{\text{Sin}[\theta ]}{\text{Cos}[\theta ]^2+\text{Sin}[\theta ]^2}
\\
 -\frac{\text{Sin}[\theta ]}{\text{Cos}[\theta ]^2+\text{Sin}[\theta ]^2} & \frac{\text{Cos}[\theta ]}{\text{Cos}[\theta ]^2+\text{Sin}[\theta ]^2}
\\
\end{array}
\right)\)\)

V.a

\begin{doublespace}
\noindent\(\pmb{\text{Custom2by2Inverse}[\text{varM$\_$}]\text{:=}\text{Module}[\{\text{vM}=\text{varM}, a,b,c,d,\text{Res}\},}\\
\pmb{\text{(*}\text{Let}'s \text{check} \text{if} \text{the} \text{matrix} \text{is} \text{of} \text{size} 2\text{x2}\text{*)}}\\
\pmb{\text{If}[\text{Dimensions}[\text{vM}][[1]]==2 \&\& \text{Dimensions}[\text{vM}][[2]]==2,}\\
\pmb{a=\text{vM}[[1,1]];}\\
\pmb{b=\text{vM}[[1,2]];}\\
\pmb{c=\text{vM}[[2,1]];}\\
\pmb{d=\text{vM}[[2,2]];}\\
\pmb{\text{(*}\text{The} \text{matrix} \text{is} \text{invertible} \text{if} \text{and} \text{only} \text{if} \text{ad}-\text{bc}\neq 0 \text{...}\text{*)}}\\
\pmb{\text{If}[a*d-b*c==0,}\\
\pmb{\text{Print}[\text{{``}[ERROR] The matrix is not invertible.{''}}]}\\
\pmb{];}\\
\pmb{\text{(*Use the provided equation to calculate the inverse of the 2x2 matrix*)}}\\
\pmb{\text{Res}=\frac{1}{a*d-b*c}\left(
\begin{array}{cc}
 d & -b \\
 -c & a \\
\end{array}
\right);}\\
\pmb{\text{(*}\text{Use} \text{Mathematica}'s \text{function} \text{Inverse}[] \text{to} \text{verify} \text{the} \text{calculation}\text{*)}}\\
\pmb{\text{If}[\text{Res}==\text{Inverse}[\text{varM}],}\\
\pmb{\text{Print}[\text{{``}[PASS] The calculation is EQUAL to Mathematica's function Inverse[{''}},}\\
\pmb{\text{MatrixForm}[\text{varM}],\text{{``}].{''}}],}\\
\pmb{\text{Print}[\text{{``}[FAIL] The calculation is NOT EQUAL to Mathematica's function Inverse[{''}},}\\
\pmb{\text{MatrixForm}[\text{varM}],\text{{``}].{''}}]}\\
\pmb{];}\\
\pmb{\text{(*Return the calculated inverse of vM*)}}\\
\pmb{\text{Res},}\\
\pmb{\text{(*Display an error message if the matrix size is not 2x2*)}}\\
\pmb{\text{Print}(\text{{``}[ERROR] Unfortunately, this function only works with 2x2 matrices.{''}})}\\
\pmb{];}\\
\pmb{];}\\
\pmb{\text{Custom2by2Inverse}\left[\left(
\begin{array}{cc}
 1 & 2 \\
 2 & 8 \\
\end{array}
\right)\right];}\\
\pmb{\text{MatrixForm}[\%]}\)
\end{doublespace}

\noindent\(\text{[PASS] The calculation is EQUAL to Mathematica's function Inverse[}\left(
\begin{array}{cc}
 1 & 2 \\
 2 & 8 \\
\end{array}
\right)\text{].}\)

\begin{doublespace}
\noindent\(\left(
\begin{array}{cc}
 2 & -\frac{1}{2} \\
 -\frac{1}{2} & \frac{1}{4} \\
\end{array}
\right)\)
\end{doublespace}

V.b

\begin{doublespace}
\noindent\(\pmb{\text{Custom2by2Inverse}\left[\left(
\begin{array}{cc}
 0 & 2 \\
 3 & 0 \\
\end{array}
\right)\right];}\\
\pmb{\text{MatrixForm}[\%]}\)
\end{doublespace}

\noindent\(\text{[PASS] The calculation is EQUAL to Mathematica's function Inverse[}\left(
\begin{array}{cc}
 0 & 2 \\
 3 & 0 \\
\end{array}
\right)\text{].}\)

\begin{doublespace}
\noindent\(\left(
\begin{array}{cc}
 0 & \frac{1}{3} \\
 \frac{1}{2} & 0 \\
\end{array}
\right)\)
\end{doublespace}

V.c\\
Infinite expression encountered.\\
Matrix is singular.

\begin{doublespace}
\noindent\(\pmb{\text{Custom2by2Inverse}\left[\left(
\begin{array}{cc}
 2 & 0 \\
 4 & 0 \\
\end{array}
\right)\right];}\\
\pmb{\text{MatrixForm}[\%]}\\
\pmb{\pmb{\text{(*}\text{$\texttt{"}$Infinite expression }\frac{1}{0}\text{ encountered.$\texttt{"}$}\text{*)}}}\)
\end{doublespace}

\noindent\(\text{[ERROR] The matrix is not invertible.}\)

V.d

\begin{doublespace}
\noindent\(\pmb{\text{Custom2by2Inverse}\left[\left(
\begin{array}{cc}
 \text{Cos}[\theta ] & -\text{Sin}[\theta ] \\
 \text{Sin}[\theta ] & \text{Cos}[\theta ] \\
\end{array}
\right)\right];}\\
\pmb{\text{MatrixForm}[\%]}\)
\end{doublespace}

\noindent\(\text{[PASS] The calculation is EQUAL to Mathematica's function Inverse[}\left(
\begin{array}{cc}
 \text{Cos}[\theta ] & -\text{Sin}[\theta ] \\
 \text{Sin}[\theta ] & \text{Cos}[\theta ] \\
\end{array}
\right)\text{].}\)

\begin{doublespace}
\noindent\(\left(
\begin{array}{cc}
 \frac{\text{Cos}[\theta ]}{\text{Cos}[\theta ]^2+\text{Sin}[\theta ]^2} & \frac{\text{Sin}[\theta ]}{\text{Cos}[\theta ]^2+\text{Sin}[\theta ]^2}
\\
 -\frac{\text{Sin}[\theta ]}{\text{Cos}[\theta ]^2+\text{Sin}[\theta ]^2} & \frac{\text{Cos}[\theta ]}{\text{Cos}[\theta ]^2+\text{Sin}[\theta ]^2}
\\
\end{array}
\right)\)
\end{doublespace}

\end{document}
