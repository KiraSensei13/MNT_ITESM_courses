% Chapter Template

\chapter{Hypothesis and Research Questions} % Main chapter title

\label{Chapter:HypothesisandResearchQuestions}

%\subsubsection*{\color{mygray}[Chapter ready for review]}
% The system/technique/parameter X is better, for task Y, than each of its rivals Z, in dimension W.

\section{Research Hypothesis}

The rheological properties of polymer solutions along with synthesis parameters (stage velocity, voltage, dispense rate) can be amended through rheological analyses to obtain a low voltage electrospun-able, photopolymerizable and graphitizable fibers for the synthesis of carbon nano-wires with specified dimensions (diameter and length). The rheological properties of polymer solutions along with synthesis parameters are to be amended by replacing the PEO (Poly(ethylene) oxide) component within the existing polymer solutions described in Flores \cite{Flores2017} and Cardemas \cite{Cardenas2017} work. PEO is to be replaced as its only purpose is to allow the electrospinning process to take place, but not benefit is obtained from it after pyrolysis.

\section{Research Questions}

\begin{itemize}
	\item{
	Is there any evidence of carbon nano-wire fabrication though electrospun-able and pyrozable polymer solutions?
	}
	\item{
	What are the process parameters to consider/control for the fabrication processes of carbon nano-wires? 
	}
	\item{
	What rheological properties are to be controlled/tested to deliver a electrospun-able and pyrozable polymer solution?	
	}
	\item{
	Are there any efforts employed to the design of polymer solutions that can be electrospun, photopolymerized, and pyrolyzed into conducting carbon nanowires?
	}
	\item{
	Are the optimal fabrication parameters defined \cite{Cardenas2017} for the synthesis of carbon nano-wires through near-field electromechanical spinning?	
	}
	\item{
	What PEO-similar materials can be used to allow the electrospinning technique and favor the carbon nano-wire properties after pyrolysis? 
	}
\end{itemize}

%----------------------------------------------------------------------------------------
%	SECTION 1
%----------------------------------------------------------------------------------------



%-----------------------------------
%	SUBSECTION 1
%-----------------------------------

