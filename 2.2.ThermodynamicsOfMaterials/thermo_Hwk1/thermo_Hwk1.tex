

\documentclass[11pt]{article} % use larger type; default would be 10pt

\usepackage[latin1]{inputenc} % set input encoding (not needed with XeLaTeX)


\usepackage{geometry} % to change the page dimensions
\geometry{letterpaper} % or letterpaper (US) or a5paper or....
 \geometry{margin=2cm} % for example, change the margins to 2 inches all round
% \geometry{landscape} % set up the page for landscape
%   read geometry.pdf for detailed page layout information

\usepackage{graphicx} % support the \includegraphics command and options

% \usepackage[parfill]{parskip} % Activate to begin paragraphs with an empty line rather than an indent

%%% PACKAGES
\usepackage{booktabs} % for much better looking tables
\usepackage{array} % for better arrays (eg matrices) in maths
\usepackage{paralist} % very flexible & customisable lists (eg. enumerate/itemize, etc.)
\usepackage{verbatim} % adds environment for commenting out blocks of text & for better verbatim
\usepackage{subfig} % make it possible to include more than one captioned figure/table in a single float
% These packages are all incorporated in the memoir class to one degree or another...
\usepackage{pdfpages}
%%% HEADERS & FOOTERS
\usepackage{fancyhdr} % This should be set AFTER setting up the page geometry
\pagestyle{empty} % options: empty , plain , fancy
\renewcommand{\headrulewidth}{0pt} % customise the layout...
\lhead{}\chead{}\rhead{}
\lfoot{}\cfoot{\thepage}\rfoot{}

%%% SECTION TITLE APPEARANCE
\usepackage{sectsty}
\allsectionsfont{\sffamily\mdseries\upshape} % (See the fntguide.pdf for font help)
% (This matches ConTeXt defaults)

%%% ToC (table of contents) APPEARANCE\item La siguiente Tabla presenta los datos de viscosidad, $\eta$, a diferentes temperaturas a 1\,atm de presi�n:


\usepackage[nottoc,notlof,notlot]{tocbibind} % Put the bibliography in the ToC
\usepackage[titles,subfigure]{tocloft} % Alter the style of the Table of Contents
\renewcommand{\cftsecfont}{\rmfamily\mdseries\upshape}
\renewcommand{\cftsecpagefont}{\rmfamily\mdseries\upshape} % No bold!

\newcommand{\chm}{\mathrm}                                                  
\newcommand{\tab}{\hspace{5mm}}
 \newcommand{\eqrx}{\rightleftharpoons}
 \newcommand{\rrx}[2]{\:\stackrel{\!\!#1}{_{\stackrel{\textstyle \eqrx}{\scriptstyle #2}}}\:}
 \newcommand{\equil}[2]{\raisebox{-2mm}{\stackrel{\!\!#1}{\stackrel{\displaystyle\eqrx}{\,_{#2}}}}}
 \newcommand{\rx}[1]{\stackrel{\!\!#1}{-\!\!\!\rightarrow}}	
 \newcommand{\grad}{\overline{\nabla}}
 \newcommand{\sbs}{_\chm}
 \newcommand{\incom}{\rule{5mm}{3mm}}
 \newcommand{\degree}{$^\circ$}



\begin{document}
{\bfseries \noindent Q4001 Thermodynamics of Materials \hfill August, 2019\\
Homework 1\\
}
\begin{enumerate}
\item Determine which of the following systems are in equilibrium.
\begin{enumerate} 
\item A rigid insulated cylinder is divided by an insulated piston into two equal parts. One part contains oxygen at 50 \degree C and 300\,kPa, and the other part contains nitrogen at 30 \degree C and 100\,kPa. The piston is held in place by a stop.
\item Same as part a, but the stop is removed.
\item A rigid insulated cylinder is divided by a copper piston into two equal parts.
One part contains oxygen at 50 \degree C and 300\,kPa, and the other part contains nitrogen at 30 \degree C and 100 \,kPa.
\item A rigid insulated cylinder is divided by a copper piston into two equal parts. One part contains oxygen at 30 \degree C and 100\,kPa, and the other part contains nitrogen at 50 1C and 100 kPa.
\item A rigid insulated cylinder is divided by a copper piston into two equal parts. One part contains oxygen at 30 \degree C and 100 kPa, and the other part contains nitrogen at 30 \degree C and 100 kPa. A small hole is made in the piston.
\item Twenty grams of salt crystals are put in a container together with 1000 cm$^3$ of water.
\item Twenty grams of salt crystals are put in a container together with 1000 cm$^3$ of saturated saline solution.
\end{enumerate}
\item A perpetual motion machine of the first kind (PMM1) is defined as an {\em adiabatic} system for which the work in a cycle is not zero. Is the first law equivalent to the statement that a PMM1 is impossible? Explain.
\item The solar energy can be harnessed to produce electric work. A PV solar panel produces 100\,kW of electric power per day in summer, which is used to charge a storage battery. The battery loses heat to the environment at a rate of 3\,kW. How much energy is stored in the battery after 6\,h of operation?
\item The envelope of an elastic balloon exerts a pressure on its contents proportional to its area. The balloon is inflated from a volume of 10\,L and a pressure of 200\,kPa to a volume of 25 L.
\begin{enumerate}
\item Calculate the final pressure in the balloon.
\item Calculate the work of the balloon envelope.
\end{enumerate}
\item For a simple compressible system, indicate whether the following statements are true, sometimes true, or false:
\begin{enumerate}
\item The work done by a system when changes from states 1 to 2 depends on its volume.
\item When a gas expands, its energy increases.
\item The work of an adiabatic cycle is equal to 0.
\item The work in a cycle is equal to 0.
\item The energy change from states 1 to 2 depends on the work of the system.
\item When a gas is compressed in a cylinder, its energy does not change.
\end{enumerate}
\item The specific energy of a certain substance is given by $U = A + BPV$, where $A =
60$\,kJ/kg, $B = 8$, $P$ units are kPa, and $V$ is expressed in in m$^3$/kg. Find the work done by a system of this substance when its volume increases adiabatically from 1.5 MPa and 60 to 150 L.
\end{enumerate}
\vfill
Due date: Tuesday, august 20th.
\end{document}
